%TC:group comment 0 0

\documentclass[A4paper]{article}


\usepackage[T2A,T1]{fontenc}
\usepackage[utf8]{inputenc}%(only for the pdftex engine)
\usepackage[russian,greek,english]{babel}
%\RequirePackage[no-math]{fontspec}[2017/03/31]%(only for the luatex or the xetex engine)
\let\OGtableofcontents\tableofcontents
\usepackage[small]{dgruyter}
\usepackage{microtype}


\usepackage{textalpha}	% enable Greek letters
%\usepackage[doublespacing]{setspace}

%%% Bibliography

%\usepackage[round]{natbib}
\usepackage[round, sort, numbers, authoryear]{natbib}
%\usepackage[backend=biber, style=authoryear, citestyle=alphabetic, sorting=nty, natbib=true]{biblatex}
\setcitestyle{aysep={},notesep={: }}

%\bibliography{/home/georg/academia/BibTeXreferences/literature}

%%%%%%%
%% FONT (possibly to be adapted)
\usepackage{libertinus}

%%%
\usepackage{comment}

\usepackage{bbding}	% ticks and crosses (symbols)

\usepackage{combelow}	% for Romanian comma below

%longtable 
\usepackage{longtable}
%\usepackage{supertabular}
\usepackage{booktabs}
\usepackage[labelsep=colon,tableposition=top,justification=raggedright,labelfont=bf,format=hang]{caption}
\usepackage{subcaption}

 \usepackage{vwcol}

\usepackage{pdflscape}




%% fix @tablefont
\makeatletter
\def\@tablefont{\normalfont}
\makeatother


% *** Trees
\usepackage{tikz}
%\usepgflibrary{arrows,shapes.misc,shadows}
\usetikzlibrary{arrows,shapes.misc,shadows,shapes.multipart}
\usepackage{tikz-qtree}
\usepackage{tikz-qtree-compat}
\tikzset{every tree node/.style={align=center, anchor=north}}	% allows multiple lines in tree node (as side effect, cf. handbook)


% *** Examples
\usepackage{expex}
\lingset{aboveglftskip=0cm, exskip=1.3ex plus .3ex minus .3ex}	% don't italicise first line of examples



\usepackage{leipzig}	% acronym
\makeglossaries
%%%%%%%%%%%%%%%%%%
% Self-defined macros
%%%%%%%%%%%%%%%%%%

\newcommand{\citeapos}[1]{\citeauthor{#1}'s (\citeyear{#1})} %for genetive 's
\newcommand{\citeapospage}[2]{\citeauthor{#2}'s (\citeyear{#2}, #1)} %for genetive 's
\newcommand{\unt}[1]{\hbox{}$_\text{#1}$}


%%%%%%%%%%%%%%%%%%%%
%% Abbreviations
%%%%%%%%%%%%%%%%%%%%%

\newcommand{\cn}{n}	
\newcommand{\cv}{v}	
\newcommand{\png}{\textsc{png}}	% person-number-gender

\newcommand{\auth}{\text{auth}}
\newcommand{\participant}{\text{part}}

\newcommand{\demft}{\text{dem}}
\newcommand{\defft}{\text{def}}

\newcommand{\commongend}{\text{cg}}	% common gender

\newcommand{\person}{\text{person:}}
\newcommand{\numb}{\text{number:}}
\newcommand{\gend}{\text{gender:}}
\newcommand{\case}{\text{case:}}

\newacronym[sort=pers]{persn}{\textsc{pers\unt{n}}}{adnominal person}
\newacronym[plural={APCs}, sort=apc]{apc}{APC}{adnominal pronoun construction}
\newacronym[plural={BPCs},longplural={bound person constructions}, sort=bpc]{bpc}{BPC}{bound person construction}
\newacronym[sort=xnp]{xnp}{\textnormal{\emph{x}nP}}{extended nominal projection}
\newacronym[sort=png]{png}{\textsc{png}}{person-number-gender}
\newacronym[plural={PoVs}, longplural={points of variation}, sort=pov]{pov}{PoV}{point of variation}
\newacronym{aic}{AIC}{Akike information criterion}
\newacronym{ppdc}{PPDC}{personal pronoun-demonstrative construction}

\newacronym{tng}{TNG}{Trans-New Guinea}
%\newacronym{tap}{TAP}{Timor-Alor-Pantar}
\newacronym{pn}{PN}{Pama-Nyungan}
\newacronym{ie}{IE}{Indo-European}

%\glsunset{tng}
%\glsunset{tap}
%\glsunset{pn}
%\glsunset{ie}
%%%%%%%%%%%%%%%%%%%%%%%
% Glosses
%%%%%%%%%%%%%%%%%%%%%%%%


\newleipzig{auth}{auth}{author}
\newleipzig{part}{part}{participant}

\newleipzig{lig}{lig}{ligature}			% Bilua
\newleipzig{pros}{pros}{prospective marker}	% Bilua
\newleipzig{sit}{sit}{situation-change marker}	% Bilua
\newleipzig{plz}{plz}{personaliser}	% Gorokan languages

\newleipzig{prtcl}{prtcl}{particle}	% 

\newleipzig{inc}{inc}{inceptive}	
\newleipzig{trn}{trn}{transitivising suffix}
\newleipzig{spec}{spec}{specific}
\newleipzig{lda}{lda}{locative-directional-ablative}
\newleipzig{proxart}{prxart}{proximate article}		% Basque
\newleipzig{todpst}{todpst}{today's past}		% 
\newleipzig{anaph}{anaph}{anaphoric pronoun}
\newleipzig{rep}{rep}{reported speech particle}
\newleipzig{merg}{merg}{merged}
\newleipzig{inch}{inch}{inchoative}
\newleipzig{ant}{ant}{anterior}
\newleipzig{emph}{emph}{emphatic}
\newleipzig{cf}{cf}{counterfactual}
\newleipzig{addr}{addr}{addressee}
\newleipzig{common}{cg}{common gender}
\newleipzig{compel}{compel}{compellative}
\newleipzig{recpst}{recpst}{recent past}
\newleipzig{lnk}{lnk}{adnominal linker}
\newleipzig{ds}{ds}{different subject}
\newleipzig{ss}{ss}{same subject}
\newleipzig{pref}{pref}{prefix}	% East Geshiza
\newleipzig{nact}{nact}{non-actual}	% East Geshiza
\newleipzig{ass}{ass}{assertive}	% Menya
\newleipzig{dso}{dso}{dissociative}
\newleipzig{gvn}{gvn}{given}
\newleipzig{cert}{cert}{certainty of assertion}
\newleipzig{hum}{hum}{human}
\newleipzig{tri}{tri}{trial/paucal}
\newleipzig{fpron}{FPron}{focus pronoun}
\newleipzig{spron}{SPron}{subject pronoun}
\newleipzig{unmark}{u}{unmarked}
\newleipzig{softmut}{softmut}{soft mutation}
\newleipzig{contr}{contr}{contrastive}
\newleipzig{redup}{redup}{reduplication}
\newleipzig{fin}{fin}{clause-final verb}
\newleipzig{nfin}{nfin}{non-final marker}
\newleipzig{ag}{ag}{agent}
\newleipzig{act}{act}{actor}
\newleipzig{npst}{npst}{non-past}
\newleipzig{mod}{mod}{modal discourse enclitic (East Geshiza)}
\newleipzig{inv}{inv}{inverse}
\newleipzig{cnt}{cnt}{continuous}  % Kokota, =durative?
\newleipzig{nf}{nf}{non-feminine}
\newleipzig{ncl}{ncl}{noun class marker}
\newleipzig{tam}{tam}{tense aspect mood marker}
\newleipzig{fv}{fv}{final vowel}
\newleipzig{habit}{habit}{habitual}
\newleipzig{cl}{cl}{clitic}
\newleipzig{generic}{gener}{generic}
\newleipzig{inan}{inan}{inanimate}
\newleipzig{stvzr}{stvzr}{stativiser}
\newleipzig{exclam}{exclam}{exclamation}
\newleipzig{ill}{ill}{illative}
\newleipzig{infl}{infl}{inflection}
\newleipzig{real}{real}{realis}
\newleipzig{postess}{postess}{postessive}
\newleipzig{prev}{prev}{preverb}
\newleipzig{nfut}{nfut}{non-future}
\newleipzig{aor}{aor}{aorist}
\newleipzig{char}{char}{characteristic}
%\newleipzig{real}{real}{realis}



\begin{comment}
\newcommand{\lig}{\glossformat{lig}}		% ligature (Obata2003)
\newcommand{\nonfem}{\glossformat{nonf}}	% non-feminine (Obata2003)
\newcommand{\pros}{\glossformat{pros}}	% prospective marker (Obata2003)
\newcommand{\situation}{\glossformat{sit}}		% situation-change marker (Obata2003)

\newcommand{\piv}{\glossformat{piv}} 	% pivotal marker (Yagaria, Renck1975)
\newcommand{\dln}{\glossformat{dln}}	% delineator (Fore, Scott 1978)
\newcommand{\plz}{\glossformat{plz}}	% personaliser (Kamano, Payne&Drew 1970)

\newcommand{\pers}{\glossformat{pers}}		% personal article
\newcommand{\inc}{\glossformat{inc}}		% inceptive
\newcommand{\trn}{\glossformat{trn}}		% transitivising suffix
\newcommand{\spec}{\glossformat{spec}}		% specific
\newcommand{\lda}{\glossformat{lda}}		% locative-directional-ablative
\newcommand{\prof}{\glossformat{pro}} 		% pro-form (for oblique pro-form in Vaeakau)

\newcommand{\real}{\glossformat{real}}		% realis (Schapper, TAP volume 2014)
\newcommand{\nfin}{\glossformat{nfin}}		% non-final (Kratichvil, TAP volume 2014)

\newcommand{\prev}{\glossformat{prev}}		% preverb in Abkhaz (Hewitt 1989)
\newcommand{\hum}{\glossformat{hum}}		% human in Abkhaz (Hewitt 1989), also Koromfe (Rennison 1997)

\newcommand{\longdet}{\glossformat{long.det}}	% long determiner (Koromfe, Rennison 1997)
\newcommand{\postpos}{\glossformat{postpos}}	% postposition (Koromfe, Rennison 1997)
\newcommand{\deict}{\glossformat{deict}}	% deictic (Koromfe, Rennison 1997), but also Keesing
\newcommand{\disjpron}{\glossformat{disjp}}	% disjunctive pronoun (Koromfe, Rennison 1997)


\newcommand{\plu}{\glossformat{plu}}	% plural article in Kwaio, Keesing

\newcommand{\indefobj}{\glossformat{indef.obj}}	% indefinite object conjugation Hungarian

\newcommand{\cf}{\glossformat{cf}}		% counterfactual
\newcommand{\emphat}{\glossformat{emph}}	% emphatic
\newcommand{\tod}{\glossformat{tod}}		% today's (past)

\newcommand{\nonpst}{\glossformat{npst}}		% non-past (e.g. Warlpiri)

\newcommand{\incompl}{\glossformat{incompl}}	%incompletive aspect (-C) in Nigerian Pidgin (Faraclas 1996)

\newcommand{\ant}{\glossformat{ant}}	% anterior (Pitjantjatjara)
\newcommand{\merg}{\glossformat{merg}}	% merged (Pitjantjatjara)
\newcommand{\cont}{\glossformat{cont}}	% merged (Pitjantjatjara)
\newcommand{\anaph}{\glossformat{anaph}}	% anaphoric (Pitjantjatjara)
\newcommand{\inch}{\glossformat{inch}}	% inchoative
\newcommand{\rep}{\glossformat{rep}}	% reported speech particle (Pitjantjatjara)

\newcommand{\pls}{\glossformat{pls}}	% plural subject agreement (Kuku Yalanji)

\newcommand{\FM}{FM}	% father's mother (Kuku Yalanji, Patz 2002)

\newcommand{\g}{\glossformat{g}}	% gender class 1 (Supyire, Carlsson 1994)

\newcommand{\fpst}{\glossformat{fpast}} % far past (Usan, Reesink 1987)

\newcommand{\tam}{\glossformat{tam}}	% Tense Aspect Marker
\newcommand{\conj}{\glossformat{conj}}	%conjunction

\newcommand{\proxart}{\glossformat{proxart}}	% proximate article in Basque
\newcommand{\cl}{\glossformat{cl}}		% clitic in greek

\newcommand{\lnk}{\glossformat{lnk}}		% (adnominal) linker morphemes

\newcommand{\supl}{\glossformat{supl}}		% superlative

\newcommand{\efoc}{\glossformat{efoc}}		% EFOC - focus marker from \emph{heo} paradigm (Lavukaleve)
\newcommand{\psnv}{\glossformat{psnv}}		% PSNV - presentative (verb suffix) (Lavukaleve)
\newcommand{\pn}{\glossformat{pn}}		% PN - demonstrative pronoun (from \emph{foia} paradigm) (Lavukaleve)
\newcommand{\ntrldist}{\glossformat{ntrl}}		% NTRL - demonstrative pronoun (from \emph{foia} paradigm) (Lavukaleve)
\newcommand{\moddem}{\glossformat{mod}}		% demonstrative modifier from \emph{hoia} paradigm (Lavukaleve)
\newcommand{\succsv}{\glossformat{succ}}		% successive verb suffix (Lavukaleve)
\newcommand{\purpsv}{\glossformat{purp}}		% purposive verb suffix (Lavukaleve)

\newcommand{\dynam}{\glossformat{dynam}}	% dynamic (Abkhaz, George Hewitt p.c.)

\newcommand{\habit}{\glossformat{habit}}	% habitual (Kobon)

\newcommand{\common}{\glossformat{c}}		% common gender (e.g. Khoekhoe)
\newcommand{\compellative}{\glossformat{compel}}		% compellative (e.g. Khoekhoe)
\newcommand{\recpst}{\glossformat{recpst}}	% recent past (e.g. Khoekhoe)
\newcommand{\authgloss}{\glossformat{auth}}	% author (Khoekhoe article)
\newcommand{\addrgloss}{\glossformat{addr}}	% addressee (Khoekhoe article)

\newcommand{\infl}{\glossformat{infl}}	%inflection (Wari')
\newcommand{\rpp}{\glossformat{rp/p}}	% realis past/present (Wari')

\newcommand{\pco}{\glossformat{pco}}	% perfective converb (Kambaata, Treis 2008)

\newcommand{\src}{\glossformat{src}}	% source (Imonda, Seiler 1985)

\newcommand{\softmut}{\glossformat{sm}}		% soft mutation (Welsh)


\newcommand{\aorist}{\glossformat{aor}}		% aorist (e.g. Georgian)
\newcommand{\pv}{\glossformat{prev}}		% preverb (Georgian)

\newcommand{\ncl}{$^{\glossformat{ncl}}$} % noun class (Bantu)
\end{comment}




\begin{document}

  \articletype{}

%  \author*[1]{...}
%  \author[2]{...}
  \author[1]{Georg F.K. Höhn} 
  \runningauthor{G.F.K. Höhn}
  \affil[1]{Georg-August-University Göttingen}
%  \affil[2]{...} 
  \title{A word order typology\\of adnominal person}
  \runningtitle{Typology of adnominal person}
  \subtitle{Supplementary Material S3}
%  \abstract{This paper investigates crosslinguistic variation in the expression of nominal person in expressions like English ``we linguists'' based on a survey of 114 languages, focusing on word order. Two general subtypes are distinguished according to whether nominal person is expressed by an independent pronoun as in English or by a morphologically dependent marker. Prenominal adnominal pronouns are the most common type of adnominal person marking overall, while the morphologically dependent markers are predominantly postnominal (or phrase-final). The order of nominal person marking relative to its accompanying noun is shown to interact with indicators of head-directionality (dependent genitives, VO/OV-order, adposition directionality) and with the directionality of demonstrative modifiers (prenominal/postnominal) using mixed linear models. Possible theoretical implications are discussed concerning variation in the encoding of nominal person as head or phrasal modifier and its co-categoriality with demonstratives. \\
% TC:ignore
% TC:endignore

%  \keywords{adnominal pronoun constructions, pronominal determiners, nominal person, headedness, demonstratives}
%  \classification[PACS]{...}
%  \communicated{...}
%  \dedication{...}
%  \received{...}
%  \accepted{...}
  \journalname{Linguistic Typology}
%  \journalyear{}
%  \journalvolume{}
%  \journalissue{}
  \startpage{1}
%  \aop
%  \DOI{...}

\maketitle

%TC:group comment 0 0

%\OGtableofcontents

%\tableofcontents

 
% Acronyms: \printacronyms[style=inline]


\section{Notes on postnominal and ambidirectional \glspl{apc} and \glspl{bpc}}

Table~\ref{table:postAPC} lists languages with postnominal \glspl{apc} with their word order properties. Several examples were already presented in Section 2.3 of the main paper.  

\begin{table}[htb!]
\caption{Word order properties of languages with postnominal pronouns\label{table:postAPC}}
\centering
 \begin{tabular}{llllll}
    \textbf{Language}  & \textbf{Classification}         & \textbf{Adpos.} & \textbf{Genit.} 	& \textbf{WO}	& \textbf{Demonstr.}   \\
    \midrule
  Warlpiri 	& Western \gls{pn}		& post 	& GenN 	& NC & NDem \\ 
  Lavukaleve 	& Lavukaleve 			& post 	& GenN 	& OV & NDem \\ 
  Maybrat 	& Maybrat 			& pre 	& NoDom & VO & NDem \\ 
  Savosavo 	& Savosavo 			& post 	& GenN 	& OV & DemN \\ 
%  Fore 		& \gls{tng}, Fore-Gimi 		& post 	& GenN 	& OV & DemN \\ 
  Yagaria 	& \gls{tng}, Siane-Yagaria 	& post 	& GenN 	& OV & DemN \\ 
  Amele 	& \gls{tng}, Mabuso 		& post 	& GenN 	& OV & NDem \\ 
  Adang 	& \gls{tng}, Alor-Pantar 	& post 	& GenN 	& OV & NDem \\ 
  Kaera 	& \gls{tng}, Alor-Pantar 	& post 	& GenN 	& OV & NDem \\ 
  Kamang 	& \gls{tng}, Alor-Pantar 	& NoDom & GenN 	& OV & NDem \\ 
  Sawila 	& \gls{tng}, Alor-Pantar 	& NoAdpos & GenN & OV & NDem \\ 
  Wersing 	& \gls{tng}, Alor-Pantar 	& NoAdpos & GenN & OV & NDem \\ 
  Western Pantar & \gls{tng}, Alor-Pantar 	& post 	& GenN 	& OV & NDem \\ 
  Papuan Malay 	& Austron., Malayo-Sumbawan 	& pre 	& GenN 	& VO & NDem \\ 
  Kalaallisut 	& Eskimo			& post 	& GenN 	& OV & mixed \\ 
  East Geshiza 	& Sino-Tibetan, Burmo-Qiangic 	& post 	& GenN 	& OV & DemN \\ 

    \end{tabular} 
\end{table}

\citet[257]{fortescue1984} presents Kalaallisut as having prenominal \glspl{apc} based on (\nextx b), which, however, involves a participial construction and is therefore presumably not ad-\emph{nominal}. I classify the language as having postnominal \gls{apc} on the basis of examples like (\nextx a) with a plain nominal. 

\pex Kalaallisut
\a
\begingl
\gla kalaalli-t \textbf{uagut}//
\glb Greenlander-\Abs{}.\Pl{} we//
\glft `we Greenlanders'\\{\citep[after][110; gloss extrapolated]{fortescue1984}}//
\endgl
\a
\begingl
\gla \textbf{uagut} kalaali-u-sugut//
\glb we Greenlander-be-1\Pl{}.\Ptcp{}//
\glft `we Greenlanders' \\{\citep[after][257]{fortescue1984}}//
\endgl
\xe

%For Warlpiri, \citet[317]{hale1973} notes that ``noun phrases of the form /\ng{}arka \ng{}atju/ `I man', /yapa \ng{}atju/ `I person' [\ldots] are possible, albeit rare, in actual usage''.
A postnominal \glspl{apc} from East Geshiza is illustrated in (\nextx). Apart from several postnominal \gls{apc} examples the grammar contains a single example of a prenominal \gls{apc} \citep[388, (5.74)]{honkasalo2019}, which incidentally also contains a numeral modifier. I still classify the language as having postnominal \glspl{apc}, since this is clearly the default option (Sami Honkasalo, pers. comm.) and the significance of the exceptional datapoint is currently unclear.
 
\ex East Geshiza\\
\begingl
\gla {}[b\ae{} \textbf{\ng\ae{}=\textltailn{}ə}=t\super{h}ə] `mbəzli' d-ə-jo\ng{}.//
\glb \phantom{[}Tibetan 1=\Pl{}=\Top{} ritual.tripod \Pref{}-\Nact{}-say.1//
\glft `We Tibetans call it \emph{mbəzli} (ritual tripod).'\\
\citep[400, 5.115]{honkasalo2019}\label{ex:geshizappdc}//
\endgl
\xe


The status of ambidirectional \glspl{apc} is not entirely clear due to limited available information on the relevant constructions, but the seven languages in Table~\ref{table:ambiAPC} appear to allow both pre- and postnominal pronouns, with the available description or data providing no clear indication whether one of them behaves as a default. 

\begin{table}[htb!]
\centering
\caption{Languages with ambidirectional \glspl{apc}\label{table:ambiAPC}}
\begin{tabular}{llllll}
\textbf{Language} & \textbf{Genus}      &  \textbf{Adpos.} & \textbf{Genit.} & \textbf{WO} & \textbf{Demonstr.}   \\
  \midrule
Pitjantjatjara 	& Western \gls{pn} 	& post 		& GenN 	& OV 		& NDem \\ 
Guugu Yimidhirr & Northern \gls{pn} 	& unclear 	& NoDom & OV 		& DemN \\ 
Kuku-Yalanji 	& Northern \gls{pn}	& pre 		& NoDom & OV 		& mixed \\ 
Imonda 		& Border 		& post 		& GenN 	& OV 		& mixed \\ 
Usan 		& North Adelbert 	& post 		& GenN 	& OV 		& NDem \\ 
Kobon 		& \gls{tng}, Kalam-Kobon & post 	& GenN 	& OV 		& NDem \\ 
Katu 		& Katuic 		& pre 		& NGen 	& unclear 	& NDem \\ 

\end{tabular}
\end{table}


Katu is the only language in the sample with evidence of pre- and postnominal pronouns occurring simultaneously, cf.\ (\nextx c).\footnote{For other instances of double marking of \gls{persn} see Table~\ref{table:cliticPerson} below.} Unfortunately, \citet{costello1969} does not discuss what determines the choice between the options in (\nextx).

\pex Katu
\a \begingl
\gla manuih \textbf{yi}//
\glb people we//
\endgl
\a 
\begingl
\gla \textbf{yi} manuih//
\glb we people//
\glft `we people'//
\endgl
\a
\begingl
\gla \textbf{yi} adi anó \textbf{yi}//
\glb we older.brother younger.brother we//
\glft `we older and younger brothers'\\
\citep[28, (35--37)]{costello1969}//
\endgl
\xe

For some other languages, information structure might influence the choice between pre- or postnominal \glspl{apc}.
For Kuku Yalanji, \citet[202]{patz2002} suggests %a connection to anaphoric versus new reference. 
that prenominal pronouns ``can be assumed to have anaphoric or definite reference'', while postnominal pronouns ``normally [establish] a new reference''. 
%In Pitjantjatjara, all \glspl{apc} provided in the grammar involve the third person singular pronoun \emph{palu\underline{r}u}.
Referring to personal communication from Paul Eckhert, \citet[34]{bowe1990} suggests that prenominal (as opposed to the unmarked postnominal) demonstratives in Pitjantjatjara ``seem to imply contrast''. Considering the translation of (\ref{ex:pitjaAPCpre}), prenominal \emph{palu\underline{r}u} appears to have the same effect. While one might speculate that postnominal \glspl{apc} also represent the unmarked order, in the absence of explicit confirmation I classify Pitjantjatjara as having ambidirectional \glspl{apc} for now. %For the demonstratives, and by extension presumably the pronouns, Bowe argues that the postnominal ones 
%Other pronouns can also be accompanied by \emph{palu\underline{r}u} (\anextx). Its singular number features seem to be neutralised in this case. 

\pex Pitjantjatjara
\a
\begingl
\gla {}[Minyma \textbf{palu\underline{r}u}] ngayu-nya nya-ngu//
\glb \phantom{[}woman 3\Sg{}.\Nom{} 1\Sg{}.\Acc{} see-\Pst{}//
\glft `The woman saw me.' \\{\citep[31, (110)]{bowe1990}}//
\endgl
\a \label{ex:pitjaAPCpre}
\begingl
\gla {}[\textbf{Palu\underline{r}u} wati nyara wa\underline{r}a-ngku] mutaka palya-nu//
\glb \phantom{[}3\Sg{}.\Nom{} man \Dem.3 tall-\Erg{} car fix-\Pst{}//
\glft `The tall man over there (in contrast to the other one) fixed the car.' \\{\citep[after][34, (114)]{bowe1990}}//
\endgl
\xe

Finally, \citet[53f., 167]{reesink1987} describes \textsc{postnominal} \glspl{apc} in Usan (\nextx a) as indicative of contrastive topics, topic shift or emphasis on topics. However, prenominal \glspl{apc} are not specifically addressed in this grammar at all. Example (\nextx b) occurs as part of a discussion on sentence connectors and the second sentence picks up the object of the first clause as \gls{apc} subject with a prenominal \gls{apc}, suggesting compatibility with topic shift, too. The significance of \gls{apc} order in Usan thus remains unclear.  %Admittedly, it is not clear if the prenominal \gls{apc} in (\nextx b) could not also be construed as contrastive. 

\pex Usan
\a \begingl
\gla eâb igim-ine ne [tain \textbf{wo}] yâ-nâmb wogub\ldots//
\glb cry.\Ss{} be-1\Sg.\Ds{} and \phantom{[}father he me-hit.\Ss{} cease.\Ss{}//
\glft `I was crying and my father he hit me and then\ldots' \\
\citep[after][167, (99)]{reesink1987}//
\endgl
\a 
\begingl
\gla \ldots{} in bo [\textbf{an} wau moi] qomon gâs ende ig-oun. Ne [\textbf{an} munon moi e]\ldots//
\glb {} we again \phantom{[}you.\Pl{} child unmarried custom like thus be-1\Pl.\Prs{} and \phantom{[}you.\Pl{} man unmarried \Dem.\Prox{}//
\glft `\ldots we in turn live like the customs of you young men. And you young men here\ldots'\\
\citep[initial part of][190f., (41)]{reesink1987}\label{ex:usanppdc}//
\endgl
\xe 

%\ex
%\begingl
%\gla Ka kunyu panya \textbf{palu\underline{r}u} pula ngari-ra tjiri\underline{r}pi-ri-ngu//
%\glb and \Rep{} \Anaph{} 3\Sg{}.\Nom{} 3\Du{}.\Nom{} lie-\Ant(\Merg) day-\Inch-\Pst{}//
%\glft `And the two of them lay down until morning.' \\{\citep[48, (179)]{bowe1990}}//
%\endgl
%\nomenclature[G]{\inch}{inchoative}%
%\nomenclature[G]{\ant}{anterior}%
%\nomenclature[G]{\merg}{merged (Pitjantjatjara, \citealp{bowe1990})}%
%\nomenclature[G]{\anaph}{anaphoric}%
%\nomenclature[G]{\rep}{reported speech particle (Pitjantjatjara, \citealp{bowe1990})}
%\xe


Morphologically bound \gls{persn} marking is attested in the languages in Table~\ref{table:cliticPerson}. 
Note that Bilua (\nextx) %\citep[6, 272]{obata2003} 
and Windesi Wamesa (\anextx) are the only VO languages with postnominal \glspl{bpc}. VO order in Bilua probably results from contact with Austronesian languages \citep[323]{terrill2011solomoncontact}, see Figure~3 in the main paper. Being Austronesian, VO order is unsurprising for Windesi Wamesa, but it is the only Austronesian language in the sample with \glspl{bpc}. It seems no coincidence that it is spoken in the same area of intense contact between Austronesian and non-Austronesian languages as Papuan Malay, the only Austronesian language in the sample with postnominal \glspl{apc} (Table~\ref{table:postAPC}) and itself an important contact language for Windesi Wamesa (\citealp[68f.]{gasser2014}; \citealp[27--33]{kluge2017}), see Figure~2 in the main paper.
%Non-third person marking is presumably not available in the singular, however (Emily Gasser, p.c.).

\ex Bilua\label{ex:biluabpc}\\
\begingl
\gla enge=ko visi=\textbf{nga}//
\glb 1\Pl.\Excl=3\Sg.\F{} younger.sibling=2\Sg//
\glft `you who are our younger sister'\\
\citep[103, (7.116)]{obata2003}//
\endgl
\xe

\ex Windesi Wamesa\label{ex:windesibpc}\\
\begingl
\gla sinitu=pa-\textbf{tata}//
\glb person=\Det{}-1\Pl{}.\Incl{}//
\glft `we people' \\{\citep[144, (3.46)]{gasser2014}}//
\endgl
\xe

%seven from Melanesia, representing five independent families (Austronesian, Angan, Central Solomons, Sepik, Trans-New Guinea), two from southern Africa (Khoekhoe, Khwe) and North America (Mi'kmaq, Classical Nahuatl) respectively and one from Europe (Basque). 
%Many of these \gls{persn} markers also indicate number and gender and are sometimes called \gls{png}-markers in the literature. 



\begin{table}[htb!]
\centering
\caption{Languages with \glspl{bpc}\label{table:cliticPerson}\\$^a$: also has prenominal \glspl{apc}, $^b$: also has postnominal \glspl{apc}, $^c$ allows prenominal pronouns in \glspl{bpc}.}
\begin{tabular}{llllll}
    \textbf{Language}  & \textbf{Classification}         & \textbf{Adpos.} & \textbf{Genit.}	& \textbf{WO} & \textbf{Demonstr.}   \\
  \midrule
\multicolumn{2}{l}{\emph{prenominal \glspl{bpc}}}\\    
    \midrule
  Classical Nahuatl 	& Aztecan 		& NoAdpos & unclear 	& NC & mixed \\ 
  Moskona$^{ac}$ 	& East Bird's Head 	& pre 	  & GenN 	& VO & NDem \\ 
    \midrule
\multicolumn{2}{l}{\emph{postnominal \glspl{bpc}}}\\    
    \midrule
  Menya$^c$ 	& Angan, Nuclear Angan 		& post 	& GenN & OV & NDem \\ 
  Bilua$^{ac}$ 	& Bilua 			& post 	& GenN & VO & DemN \\ 
  Alamblak 	& Sepik Hill 			& post 	& GenN & OV & DemN \\ 
  Fore 		& \gls{tng}, Fore-Gimi 		& post 	& GenN & OV & DemN \\ 
  Hua 		& \gls{tng}, Siane-Yagaria 	& post 	& GenN & OV & DemN \\ 
  Yagaria$^b$ 	& \gls{tng}, Siane-Yagaria 	& post 	& GenN & OV & DemN \\ 
  Windesi Wamesa & Austronesian, Oceanic	& pre 	& GenN & VO & NDem \\ 
  Mi’kmaq$^c$ 	& Algonquian 			& NoDom & GenN & NC & DemN \\ 
  Khoekhoe (Nama)$^c$ & Khoe-Kwadi 		& post 	& GenN & OV & DemN \\ 
  Khwe/Kxoe$^c$ & Khoe-Kwadi	 		& post 	& GenN & OV & DemN \\ 
  Basque$^c$	& Basque 			& post 	& GenN & OV & NDem \\ 

\end{tabular}
\end{table}


Six languages, marked by $^c$ in Table~\ref{table:cliticPerson}, also seem to permit an additional prenominal pronoun(-like) \gls{persn}-marking in \glspl{bpc} (for further discussion see Sections~5.1 and 5.3 in the main paper). For the three \gls{bpc} languages marked with $^a$ and $^b$ in Table~\ref{table:cliticPerson} there is some evidence for independent \glspl{apc} as well.

\gls{persn} in Moskona is marked by prefixes on ``[g]eneric nouns which denote humans'' \citep[219]{gravelle2010}, see (\nextx a). Adjectival (and numeral) modifiers also seem to show person-number agreement \citep[127]{gravelle2010}; see (\nextx b). Additionally, \citet[224]{gravelle2010} shows examples containing only a prenominal \gls{apc} without the prenominal \gls{persn} marker (\nextx c). They generally contain proper names, so maybe those are incompatible with Moskona's \gls{persn} prefixes.

\pex Moskona
\a
\begingl
\gla \textbf{mi}-osnok mi-en-ah-miy,  mi-en-ot jig miyes + mi-er tofi.//
\glb 1\Pl{}-person 1\Pl-\Dur{}-lie-water 1\Pl-\Dur{}-stand \Loc{} clothes 1\Pl{}-wear hat//
\glft `we people bathe, wear clothes, wear hats\ldots' (stand in clothes = wear clothes)\\
\citep[after][344, (2)]{gravelle2010}//
\endgl
\a \begingl
\gla \textbf{i}-osnok \textbf{i}-ofogo//
\glb 3\Pl-person 3\Pl-evil//
\glft `evil people'\\
\citep[194, (26a)]{gravelle2010}//
\endgl
\a
\begingl
\gla \textbf{eri} Mosmir//
\glb they Maybrat//
\glft `Maybrat people/tribe'\\
\citep[224, (43b)]{gravelle2010}//
\endgl
\xe

Independently of the postnominal \gls{bpc} mentioned above (\ref{ex:biluabpc}), Bilua also seems to allow prenominal \glspl{apc} (\nextx). The ligature marker ``occurs only between morphemes which belong to the same phrase'' \citep[79]{obata2003}, suggesting that (\nextx) does not involve apposition of two distinct noun phrases.

\ex Bilua\label{ex:biluaPreAPC}\\
\begingl
\gla \textbf{enge}=a saidi//
\glb 1\Pl{}.\Excl{}=\Lig{} family//
\glft `we, family'\\{\citep[79, (7.10)]{obata2003}}//
\endgl
%\nomenclature[G]{\lig}{ligature \citep{obata2003}}
\xe%

Finally, Yagaria has postnominal \glspl{bpc} (\nextx a), but also uses postnominal pronouns ``in focused phrases [\ldots], especially in transitive clauses where the marking of the subject is obligatory'' \citep[17]{renck1975}. The construction is also available with focused objects, see (\nextx b), making an analysis of the postnominal pronoun as a resumptive less likely, since the noun \emph{ve} is not left-peripheral here. %\footnote{While Renck's examples only involve third person pronouns, no person restriction is mentioned. Yagaria also has bound person marking, see section~\ref{ssec:Gorokan} for non-third person examples of that type.} 
%\citet[181]{renck1975} suggests that the postnominal pronominal indicates focus on the noun phrase ``somewhat in the same way as the pivotal marker'', which is in complementary distribution with adnominal pronouns in Yagaria and possibly also in Fore (\citealp[103]{scott1978}). The pivotal marker is discussed in Section~\ref{ssec:Gorokan}.
%\citet[90ff.]{deswart2007}
% some further comment?

\pex Yagaria
\a
\begingl
\gla {}[Ovu-\textbf{da}] ma-lo' bei-d-u-e//
\glb \phantom{[}Ovi-I this-\Loc{} live-\Pst{}-1.\Sg{}-\Ind{}//
\glft `I, Ovu, am here.'//
\endgl
%\a
%\begingl
%\gla \textbf{yale} \textbf{pagaea} gayale hae-d-a-e//
%\glb people they pig shoot-\Pst{}-3\Pl{}-\Ind{}//
%\glft `The people shot the pig.'\trailingcitation{\citealp[17]{renck1975}}//
%\endgl
%\a 
%\begingl
%\gla \textbf{ve} \textbf{agaea} o-d-i-e//
%\glb man he come-\Pst{}-3\Sg{}-\Ind{}//
%\glft `The man came.'\\{\citep[17]{renck1975}}//
%\endgl
%\a\label{ex:yagariaAPC.Obj}
\a
\begingl
\gla dagaea [ve \textbf{agaea}] $\emptyset$-begi-d-u-e//
\glb I \phantom{[}man he him-hit-\Pst{}-1\Sg{}-\Ind{}//
\glft `I hit the man.' \\{\citep[18f.]{renck1975}}//
\endgl
\xe

Closely related Fore might also involve postnominal \glspl{apc}, although \citet[100]{scott1978} notes the possibility of an intonational break before the ``pronominal copy'' indicated by the comma in (\nextx). Since a resumptive analysis cannot be excluded at least for this example, I treat Fore as only having postnominal \glspl{bpc} for now, cf.\ (5) in the main paper.\footnote{Classifying Fore as language with postnominal \glspl{apc} would only strengthen the tendencies observed in the next section.}

\ex Fore\\
\begingl
\gla {}[teméni'-N a-pa:'], [\textbf{áe'}] kana-y-e//
\glb Temeni-\Obl{} his-father he come-he-\Ind{}//
\glft `Temeni's father is coming.'\\{\citep[after][100, (163a)]{scott1978}}//
\endgl
\xe

\clearpage

%%%%%%%%%%%%%%%%%%%%%%%%%%%%%%%%%%%%%%%%%%%%%%%%%%%%%%%%

\section{Supporting tables}


\subsection{For Section 5.1}

\begin{table}[htb!]
\caption{Languages with prenominal \glspl{apc} and at least two indicators of head-finality\label{table:PronN-headfin}}
\centering
 \begin{tabular}{lllll}
    \textbf{Language}  & \textbf{Classification}         & \textbf{Genit.} 	& \textbf{Adpos.}	& \textbf{WO} \\
    \midrule
  Bilua & Bilua & GenN & post & VO 			\\ 
  Hungarian & Uralic, Ugric & GenN & post & VO 		\\ 
  Finnish & Uralic, Finnic & GenN & post & VO 		\\ 
  Japanese & Japanese & GenN & post & OV 		\\ 
  Korean & Korean & GenN & post & OV 			\\ 
  Evenki & Tungusic & GenN & post & OV 			\\ 
  Turkish & Turkic & GenN & post & OV 			\\ 
  Momu & Baibai-Fas & GenN & post & OV 			\\ 
  Manambu & Sepik, Ndu & GenN & post & OV 		\\ 
  Awtuw & Sepik, Ram & GenN & post & OV 		\\ 
  Chitimacha & Chitimacha & GenN & post & OV 		\\ 
  Kannada & Dravidian & GenN & post & OV 		\\ 
  Malayalam & Dravidian & GenN & post & OV 		\\ 
  Tamil & Dravidian & GenN & post & OV 			\\ 
  Kashmiri & \gls{ie}, Indic & GenN & post & OV 	\\ 
  Marathi & \gls{ie}, Indic & GenN & post & OV 		\\ 
  Punjabi & \gls{ie}, Indic & GenN & post & OV 		\\ 
  Supyire & Niger-Congo, Senufo & GenN & post & OV 	\\ 
  Lezgian & Lezgic & GenN & post & OV 			\\ 
  Abkhaz & Northwest Caucasian & GenN & post & OV 		\\ 
  Kambaata & Afroasiatic, East Cushitic & GenN & NoAdpos & OV \\ 
  Diyari & Central \gls{pn} & GenN & unclear & OV \\ 

    \end{tabular} 
\end{table}

\clearpage

\subsection{For Section 5.3}

\begin{table}[htb!]
\caption{Languages with mismatch in directionality of \glspl{apc} and demonstrative modifiers. In column \gls{ppdc}, (\Checkmark) indicates that the attested \gls{ppdc} examples involve no overt noun.\label{table:APCDemmismatch}}
\centering
 \begin{tabular}{lllp{.25\linewidth}}
    \textbf{Language}  & \textbf{Classification}        & \textbf{Alt. \Dem{} position}  &  \textbf{\gls{ppdc}}\\	%(\ref{ex:demdeviations}b)
    \midrule
    \multicolumn{2}{l}{\emph{prenominal \glspl{apc} and NDem}}\\
    \midrule
  Momu 		& Baibai-Fas & 		& \Checkmark \citealt[559]{honeyman2016}\\ 	% (9/10)
  Moskona	& East Bird's Head & 	& \Checkmark \citealt[187]{gravelle2010}\\	% (1)
  Sougb 	& East Bird's Head & 	& \Checkmark \citealt[271]{reesink2002sougb}\\  %, (13)
  Hatam 	& Hatim-Mansim & 	& \Checkmark \citealt[41, 195]{reesink1999}\\ 
  Urim 		& Urim  & 		& \Checkmark \citealt[140]{hemmilaeluoma1987}\\ 
  Mupun 	& Afroasiatic  \\						% , West Chadic 
  Gorwaa 	& Afroasiatic  &	& \Checkmark \citealt[116]{harvey2018}\\ 	% , Southern Cushitic
  Cair.\ Eg.\ Arabic & Afroasiatic & (\Checkmark) \citealt[350f.]{doss1979}\footnotemark \\ 
  Indonesian 	& Austronesian  & 	& (\Checkmark) \citealt[169]{sneddon1996}\\ 	% , Malayo-Sumbawan
  Loniu 	& Austronesian  &	& \Checkmark \citealt[100]{hamel1994}\\ 		% , Oceanic
  Tuvaluan 	& Austronesian  &	& (\Checkmark) \citealt[409]{besnier2000}\\ % , Oceanic
  Kwaio 	& Austronesian  &\\ 						% , Oceanic
  Arop-Lokep 	& Austronesian  &	& \Checkmark \citealt[255]{djernes2002}\\ 	% , Oceanic
  Cheke Holo 	& Austronesian  & \Checkmark \citealt[169]{boswell2018} & \Checkmark \citealt[165]{boswell2018}\\ 	% , Oceanic
  Hoava 	& Austronesian  &	& \Checkmark \citealt[48]{davis2003}\\ 		% , Oceanic
  Kokota 	& Austronesian  &	& \Checkmark \citealt[116]{palmer2008}\\ 	% , Oceanic
  Wari' 	& Chapacura-Wanham & 	& \\ 
  Ndyuka 	& Creoles and Pidgins & & (\Checkmark) \citealt[203, 329]{huttarhuttar1994}\\ 
  Welsh 	& \gls{ie}  & 	& \\ 			% , Celtic
  Luganda 	& Niger-Congo  & \Checkmark \citealt[41]{ashtonetal1954}	&\\ 		% , Bantu  , fn. 2
  Nkore-Kiga 	& Niger-Congo  & \Checkmark \citealt[10]{tayebwa2014} \\ 		% , Bantu
  Nzadi 	& Niger-Congo  & \Checkmark \citealt[100]{craneetal2011}\\ 		% , Bantu
  Swahili 	& Niger-Congo  & \Checkmark \citealt[35]{mpiranya2015}\\ 		% , Bantu
  Babungo 	& Niger-Congo  & \Checkmark \citealt[73]{schaub1985}\\ 		% , Wide Grassfields
  Koromfe 	& Niger-Congo  &	& \Checkmark John Rennison, pers. comm.\\ 	% , Koromfe
  Lakkia 	& Kadai  	&	& \\ 
  \midrule
  \multicolumn{2}{l}{\emph{postnominal \glspl{apc} and DemN}}\\
  \midrule
  Savosavo 	& Savosavo  &		& (\Checkmark) \citealt[86]{wegener2012}\\ 
%  Fore 		& \gls{tng}, Fore-Gimi & post & DemN \\ 
  Yagaria 	& \gls{tng}  \\ 			% , Siane-Yagaria
  East Geshiza 	& Sino-Tibetan & \Checkmark \citealt[40]{honkasalo2019} & \Checkmark (\ref{ex:geshizappdc}), \citealt[301f., 400]{honkasalo2019}\\ 		% , Burmo-Qiangic

    \end{tabular} 
\end{table}
\footnotetext{Following \citet[351]{doss1979} DemN orders in Colloquial Egyptian Arabic are highly restricted and ``constitute residuals from a previous stage during which the variation of word-order was a freer one.''}


\begin{table}[htb!]
\caption{\glspl{ppdc} in languages with \glspl{apc} with matching demonstrative directionality. (\Checkmark) indicates that attested \glspl{ppdc} contained no overt noun.\label{table:PPDCotherAPC}}
\centering
\begin{tabular}{lllp{.3\linewidth}}
    \textbf{Language}  & \textbf{Classification}    &  \textbf{\Dem{} order} & \textbf{\gls{ppdc}} \\
    \midrule
    \multicolumn{2}{l}{\emph{prenominal \glspl{apc}}}\\
    \midrule
  Japanese & Japanese 		& DemN & (\Checkmark) \citealt[214]{coulmas1982}; \citealt[153]{furuya2008}; \citealt[777]{noguchi1997} \\ 
  Korean & Korean 		& DemN & (\Checkmark) \citealt[281]{sohn1994} \\ 
  Diyari & Central \gls{pn} 	& DemN & (\Checkmark) \citealt[61, (36)]{austin1981}\\ 
  Kayardild & Tangkic 		& DemN & \Checkmark \citealt[251, (6-37)]{evans1995} \\ 
  Manambu & Sepik, Ndu 		& DemN & \Checkmark \citealt[198]{aikhenvald2008} \\ 
  Pomak &  \gls{ie}	 	& DemN & (\Checkmark) \citealt[581]{papadimitriou2008}\\ 
  Mandarin & Sino-Tibetan 	& DemN & \Checkmark \citealt[298f.]{huangetal2009}\\ 
  Hausa & Afroasiatic 		& mixed & \Checkmark \citealt[331]{jaggar2001}; \citealt[155, 371]{newman2000} \\ 
  Maori & Austronesian 		& mixed & (\Checkmark) \citealt[263f.]{bauer1997}\\ 
  Malagasy & Austronesian 	& both & \Checkmark \citealt[422]{paultravis2019} \\ 
\midrule
  \multicolumn{2}{l}{postnominal \glspl{apc}}\\
  \midrule
  Lavukaleve & Lavukaleve 	& NDem & (\Checkmark) \citealt[181, (220-1)]{terrill2003}\\ 
  Amele & \gls{tng} 		& NDem & \Checkmark (16) in main text, \citealt[210, 218]{roberts1987amele} \\ 
  Kaera & \gls{tng} 		& NDem & \Checkmark \citealt[129, (99)]{klamer2014} \\ 
  Papuan Malay 	& Austronesian 	& NDem & \Checkmark \citealt[353, (66/67)]{kluge2017} \\ 
\midrule
  \multicolumn{2}{l}{ambidirectional \glspl{apc}}\\
  \midrule  
  Kuku-Yalanji & Northern \gls{pn} 	& mixed & \Checkmark \citealt[204, (625)]{patz2002} \\
  Guugu Yimidhirr & Northern \gls{pn} 	& DemN & \Checkmark \citealt[73, (107); 157, (423); 160]{haviland1979} \\ 
  Pitjantjatjara & Western \gls{pn} 	& NDem & \Checkmark \citealt[48--51]{bowe1990} \\ 
  Usan & \gls{tng} 			& NDem & \Checkmark (\ref{ex:usanppdc}), \citealt[190, (141)]{reesink1987} \\ 
\end{tabular}  
\end{table}
  



\begin{table}[htb!]
\caption{Languages with \gls{bpc} allowing additional adnominal \gls{persn}-marking and/or \glspl{ppdc}. For (\Checkmark) the attested \glspl{ppdc} involve no overt noun.\label{table:bpcppdc}}
\centering
\begin{tabular}{lllll}
    \textbf{Language}  & \textbf{Classification}   & \textbf{\Dem{} order} & \textbf{Pron. in \gls{bpc}} & \textbf{\gls{ppdc}} \\
    \midrule   
  \multicolumn{2}{l}{\emph{Postnominal \gls{bpc}}} \\
  \midrule
  Khoekhoe & Khoe-Kwadi   & DemN & pre & \Checkmark \citealt[54]{haacke1977} \\ 
  Khwe/Kxoe & Khoe-Kwadi  & DemN & pre & \Checkmark \citealt[41, (2/3); 49, (25)]{kilianhatz2008} \\ 
  Menya & Nuclear Angan   & NDem & pre & \Checkmark \citealt[30, (58/59)]{whitehead2006} \\ 
  Alamblak & Sepik	  & DemN &     & \Checkmark \citealt[90, (149a)]{bruce1984}\\
  Bilua & Bilua  	  & DemN & pre & \Checkmark \citealt[289, (8)]{obata2003}\\ 
  Mi’kmaq & Algonquian    & DemN & pre &\\ 
  Basque & Basque  	  & NDem & pre &   \\
  \midrule
  \multicolumn{2}{l}{\emph{Prenominal \gls{bpc}}} \\
  \midrule
  Moskona & East Bird's Head & NDem & pre & \Checkmark \citealt[187, 224]{gravelle2010} \\ 

\end{tabular}
\end{table}


\newpage
%%%%%%%%%%%%%%%%%%%%%%%%%%%%%%%%%%%%%%%%%%%%%%%%%%%%%%%%

\section{Model selection for random intercept structure}

As discussed in Section 4.3 of the main text, the generalised linear mixed-effect model to be developed involves \gls{apc} directionality as dependent variable and the two fixed effects \textsc{headfin} (for head-directionality) and \textsc{demfin} (for noun-demonstrative order), see (\nextx). This additional section discusses the method used for determining the random effect structure.

\ex \textsc{APCDir} $\sim$ \textsc{headfin} + \textsc{demfin}  + \emph{random effects} \xe

Two types of random effects to consider for inclusion into the target model are phylogenetic relationships between the languages in the sample and their geographical clustering/distance. 
To asses the impact of phylogenetic distance, two levels of detail were considered, language \textsc{family} (f) and \textsc{genus} (g) following, e.g. \citet{dryer1989} and the notation in WALS \citep{wals}. In addition to either of the two phylogenetic descriptors in isolation as random intercepts, models were also fitted with \textsc{genus} nested under \textsc{family} in order to capture the phylogenetic structure.
In order to assess potential areal effects and language contact, I calculated the distances between languages based on their coordinates (spherical distances, calculated with the function function \texttt{distm}, \texttt{geosphere} package \citealt{R:geosphere}). The distance matrix between the locations of the sample languages was reduced to one and two dimensions with multidimensional scaling (\texttt{cmdscale} function). The estimates of these two dimensions for each language were used as random intercepts dist.1 and dist.2 in (42). 
The estimates of the scaling to two dimensions (2D) were coded using two random intercepts \textsc{mds1} and \textsc{mds2}. An alternative with reduction to one dimension (1D) was also considered with the estimate coded by the random intercept \textsc{mds.single}.
 
The various combinations of these random intercept structures are listed in Table~\ref{tab:randommodels}, with the dependent variable and fixed effects from (\lastx) replaced by \ldots{} for brevity.
Models marked with * did not converge and models marked with ! had a singular fit, indicating overfitting.

\begin{table}[htb!]
\centering
\caption{Random effect structures for (\lastx) \label{tab:randommodels}}
\begin{tabular}{lrrcl}
   & issues 	& \multicolumn{2}{c}{model} & random effects\\
\midrule 
a. & 		& mdir.2Dfg & = \ldots + & (1 | \textsc{mds1}) + (1 | \textsc{mds2}) + (1 | \textsc{family}/\textsc{genus}) \\
b. & \ljudge{*} & mdir.2Df  & = \ldots + & (1 | \textsc{mds1}) + (1 | \textsc{mds2}) + (1 | \textsc{family})\\
c. &  		& mdir.2Dg  & = \ldots + & (1 | \textsc{mds1}) + (1 | \textsc{mds2}) + (1 | \textsc{genus})\\
d. & \ljudge{!} & mdir.2D   & = \ldots + & (1 | \textsc{mds1}) + (1 | \textsc{mds2}) \\
e. & \ljudge{!} & mdir.1Dfg & = \ldots + & (1 | \textsc{mds.single}) + (1 | \textsc{family}/\textsc{genus})\\
f. & 		& mdir.1Df  & = \ldots + & (1 | \textsc{mds.single}) + (1 | \textsc{family})\\
g. & 	 	& mdir.1Dg  & = \ldots + & (1 | \textsc{mds.single}) + (1 | \textsc{genus})\\
h. & \ljudge{!} & mdir.1D   & = \ldots + & (1 | \textsc{mds.single})\\
i. &  		& mdir.fg   & = \ldots + & (1 | \textsc{family}/\textsc{genus})\\
j. & 		& mdir.f    & = \ldots + & (1 | \textsc{family})\\
k. & 		& mdir.g    & = \ldots + & (1 | \textsc{genus})\\
\end{tabular}
\end{table}

In order to identify a random effect structure that appropriately balances modelling detail and overfitting, the converging models are compared in an ANOVA in Table~\ref{tab:modelselect}. 


\begin{table}[htb!]
\centering
\caption{Model selection among converging models from Table~\ref{tab:randommodels} \label{tab:modelselect}}
\begin{tabular}{llrrrrrrrr}
  &	& npar & AIC & BIC & logLik & deviance & $\chi^2$ & Df & Pr($>$Chisq) \\ 
  \midrule
	& mdir.1D   & 4 & 47.414 & 57.671 & -19.707 & 39.41 &  &  &  \\ 
	& mdir.f    & 4 & 29.418 & 39.675 & -10.709 & 21.42 & 18.00 & 0 &  \\ 
\HandRight	& \textbf{mdir.g}    & 4 & \textbf{29.398} & 39.655 & -10.699 & 21.40 & 0.02 & 0 &  \\ 
	& mdir.2D   & 5 & 49.414 & 62.236 & -19.707 & 39.41 & 0.00 & 1 & 1.00 \\ 
	& mdir.1Dg  & 5 & 31.398 & 44.220 & -10.699 & 21.40 & 18.02 & 0 &  \\ 
	& mdir.1Df  & 5 & 31.418 & 44.240 & -10.709 & 21.42 & 0.00 & 0 &  \\ 
	& mdir.fg   & 5 & 31.396 & 44.218 & -10.698 & 21.40 & 0.02 & 0 &  \\ 
	& mdir.2Dg  & 6 & 33.398 & 48.784 & -10.699 & 21.40 & 0.00 & 1 & 1.00 \\ 
	& mdir.1Dfg & 6 & 33.418 & 48.804 & -10.709 & 21.42 & 0.00 & 0 &  \\ 
	& mdir.2Dfg & 7 & 35.398 & 53.348 & -10.699 & 21.40 & 0.02 & 1 & 0.89 \\ 
\end{tabular}
\end{table}

The model comparison indicates that model \texttt{mdir.g} with only \textsc{genus} as random intercept has the lowest \gls{aic}, so the analysis in the main text adopts the model in (\nextx).

\ex \textsc{APCDir} $\sim$ \textsc{headfin} + \textsc{demfin}  + (1 | \textsc{genus}) \xe


%\clearpage

\section{List of surveyed languages}

Grammars based on \citeapos{comriesmith1977} questionnaire are marked (Q). Genus information is based on the corresponding classification in WALS \citep{wals}. For languages without WALS entry (marked by *) genus is extrapolated from Glottolog \citep{glottolog4.5} or the respective grammars. The final column provides references to \gls{persn}-related discussion and/or data for each language. 


\begin{longtable}{p{.11\linewidth}p{.2\linewidth}p{.24\linewidth}p{.35\linewidth}}

\caption{List of surveyed languages \label{tab:langsample}}\\
%\begin{center}
\toprule
\textbf{Glottocode} & \textbf{Language} & \textbf{Genus} & \textbf{Main source(s)}\\
\midrule\endfirsthead
\caption[]{(continued)}\\
\toprule
\textbf{Glottocode} & \textbf{Language} & \textbf{Genus} & \textbf{Main source(s)}\\
\midrule\endhead
%\tablefirsthead{\toprule
%\textbf{Language} & \textbf{Family} & \textbf{Main source(s)}\\
%\midrule}
%\tablehead{\toprule
%\textbf{Language} & \textbf{Family} & \textbf{Main source(s)}\\
%\midrule}
%\tabletail{\multicolumn{3}{r}{\emph{continued}}\\}
%\tablelasttail{}
%\tablecaption{List of surveyed languages \label{tab:langsample}}
%\begin{mpsupertabular}{p{.23\linewidth}  p{.38\linewidth}  p{.36\linewidth}}
%\midrule
\multicolumn{4}{c}{{\emph{Afroasiatic languages (7 languages/4 genera)}}}\\
\midrule
haus1257	& Hausa 		& West Chadic 		& \citealp[63, 155, 370f.]{newman2000}; \citealp[330f.]{jaggar2001}\\
mwag1236	& Mupun 		& West Chadic 		& \citealp[172]{frajzyngier1993}\\
goro1270	& Gorwaa		& Southern Cushitic 	& \citealp[163]{harvey2018}\\
kamb1316	& Kambaata 		& East Cushitic* 	& \citealp[335]{treis2008}\\
egyp1253	& Cairene Egypt. Arabic & Semitic 	 	& (Q) \citealp[78, 80]{garygamal1982}\\
gulf1241	& Gulf Arabic 		& Semitic 	 	& (Q) \citealp[162, 165]{holes1990}\\
malt1254	& Maltese 		& Semitic 	 	& (Q) \citealp[187f., 202]{borgazzopardialexander1997}\\
\midrule
\multicolumn{4}{c}{{\emph{Australian languages (7 languages/5 genera)}}}\\
\midrule
mang1381	& Mangarrayi 		& Mangarrayi 			& (Q) \citealp[103; 203]{merlan1989}\\
%?Bilinarra & Pama-Nyungan & \citealp{meakinsnordlinger2014}\\
dier1241	& Diyari 		& Central \gls{pn} 		& \citealp[97f.]{austin1981}, \citep[102f.]{austin2013}\\
warl1254	& Warlpiri 		& Western \gls{pn} 		& \citealp[70]{reece1970}; \citealp[316f.]{hale1973}\\
pitj1243	& Pitjantjatjara 	& Western \gls{pn} 		& \citealp[49--51]{bowe1990}\\
gugu1255	& Guugu Yimidhirr 	& Northern \gls{pn} 	& \citealp[104, 156f.]{haviland1979}\\
kuku1273	& Kuku Yalanji 		& Northern \gls{pn} 	& \citealp[120f., 202f.]{patz2002}\\
%Yidin & Pama-Nyungan, Ngumbin & \citealp{dixon1977}\\
kaya1319	& Kayardild 		& Tangkic 		& \citealp[239; 251]{evans1995}; \citealt[141]{round2013}\\
\midrule
\multicolumn{4}{c}{{\emph{Austronesian languages (12 languages/3 genera)}}}\\
\midrule
%Std. Fijian & Austronesian & \citealp{palmer2017}\\
mala1537	& Malagasy 		& Barito	 	& \citealp{paultravis2019}\\
stin1234	& Indonesian 		& Malayo-Sumbawan 	& \citealp[170]{sneddon1996}\\
papu1250	& Papuan Malay 		& Malayo-Sumbawan 	& \citealp[ch. 6.2]{kluge2017}\\
loni1238	& Loniu			& Oceanic 		& \citealp[90]{hamel1994}\\
wame1241	& Windesi Wamesa 	& Oceanic* 		& \citealp[144]{gasser2014}\\
maor1246	& Maori 		& Oceanic 		& (Q) \citealp[368, 373]{bauer1993}; \citealp[262f.]{bauer1997}\\
tuva1244	& Tuvaluan 		& Oceanic 		& (Q) \citealp[392f.]{besnier2000}\\
kwai1243	& Kwaio 		& Oceanic 		& \citealp[104]{keesing1985}\\
arop1243	& Arop-Lokep		& Oceanic		& \citealp[255]{djernes2002}\\
chek1238	& Cheke Holo		& Oceanic		& \citealp[165]{boswell2018}\\
hoav1238	& Hoava 		& Oceanic		& \citealp[47]{davis2003}; \citealp{palmer2017}\\
koko1269	& Kokota		& Oceanic		& \citealp[68, 95, 116, 119, 123, 131, 137, 163, 242, 300, 305, 327, 399, 414]{palmer2008}\\
%Madurese & Austronesian, Malayo-Polynesian & \citealp{davies2010}\\
%Rapanui & Austronesian, Oceanic, Eastern Polynesian & (C) \citealp{dufeu1996}\\
%Vaeakau-Taumako & Austronesian, Oceanic, Polynesian & \citealp{naesshovdhaugen2011}\\
\midrule
\multicolumn{4}{c}{{\emph{Dravidian languages (3 languages/1 genus)}}}\\
\midrule
nucl1305	& Kannada 	& Dravidian & (Q) \citealp[205, 208f.]{sridhar1990}\\
mala1464	& Malayalam 	& Dravidian & (Q) \citealp[262f.]{asherkumari1997}\\
tami1289	& Tamil 	& Dravidian & (Q) \citealp[142, 146]{asher1985}\\
\midrule
\multicolumn{4}{c}{{\emph{Indo-European languages (27 languages/7 genera)}}}\\
\midrule
wels1247	& Welsh 		& Celtic & David Willis (pers. comm.) \\
dani1285	& Danish 		& Germanic & \citealt{johannessen2008}; \citealt{schroeter2021}\\
icel1247	& Icelandic 		& Germanic & \citealp{johannessen2008}; \citealt[11f.]{sigurdssonwood2020}\\
norw1258	& Norwegian 		& Germanic & \citealp[127, 129]{julien2005}; \citealp{johannessen2008}\\
swed1254	& Swedish 		& Germanic & \citealp[128]{julien2005}; \citealp{johannessen2008}\\
dutc1256	& Dutch 		& Germanic & \citealt[52]{corver2008vocative}\\
stan1293	& English 		& Germanic & \citealp{postal1969, delormedougherty1972, sommerstein1972, pesetsky1978, keizer2016}\\
stan1295	& German 		& Germanic & \citealp[ch. 6]{lawrenz1993}; \citealp{rauh2003, rauh2004, roehrs2005}\\
mode1248	& Std. Mod. Greek 	& Greek & \citealp[chs. 1/2]{choi2014phd}; \citealp[sec. 5]{hoehn2015unagr}\\
aspr1238	& Calabrian Greek/Greko	& Greek* &  \citealp[274]{hoehnetalICGLcalabria}\\
kash1277	& Kashmiri 		& Indic & (Q) \citealp[200]{walikoul1997}\\
mara1378	& Marathi 		& Indic  & (Q) \citealp[386]{pandharipande1997}\\
panj1256	& Punjabi 		& Indic & (Q) \citealp[228]{bhatia1993}\\
west2369	& Persian 		& Iranian & (Q) \citealp[209, 212]{mahootian1997}\\
arom1237	& Aromanian 		& Romance* & \citealp[546, 560]{hoehn2015unagr}\\
roma1327	& Romanian 		& Romance & (Q) \citealp[255, 258]{mallinson1986}, \citealp[6, 10, 20f.]{cornilescunicolae2014}\\
stan1289	& Catalan 		& Romance & (Q) \citealp[287, 290]{hualde1992catalan}; \citealt[560]{hoehn2015unagr}\\
gali1258	& Galician 		& Romance & \citealp[152, 301]{alvarezetal1986}; \citealp[560]{hoehn2015unagr}\\
port1283	& (Eur.) Portuguese 	& Romance & \citealp[555, 560]{hoehn2015unagr}\\
stan1288	& Spanish 		& Romance 	& \citealp[145]{debruyne1995}; \citealp[210f.]{choi2014phd}; \citealp[560]{hoehn2015unagr} \\
ital1282	& Italian 		& Romance 	& \citealt[202f.]{cardinaletti1994}; \citealt[559]{hoehn2015unagr}\\
nort2612	& Northern Calabrese 	& Romance* & \citealp[142]{hoehnetal2016CalabrUnagr}\\
sout2616	& Southern Calabrese 	& Romance* & \citealp[142]{hoehnetal2016CalabrUnagr}\\
russ1263	& Russian 		& Slavic & \citealp[352]{pesetsky1978}\\
bulg1262	& Bulgarian 		& Slavic & \citealp[560]{hoehn2015unagr}\\
%Macedonian & IE, Slavic, South & \citealp{friedman2002, tomic2012}\\
poma1238	& Pomak 		& Slavic* & \citealp[582]{papadimitriou2008}\\
%BCMS & IE, Slavic, South & \\
%Slovenian & IE, Slavic, South & \\
poli1260	& Polish 		& Slavic & \citealp[161]{rutkowski2002}\\
\midrule
\multicolumn{4}{c}{{\emph{Khoe-Kwadi languages (2 languages/1 genus)}}}\\
\midrule
nama1264	& Khoekhoe (Nama) 	& Khoe-Kwadi 			& \citealp[133--145]{boehm1985}; \citealp{haacke1976, haacke1977, haacke2013namamorph}; \citealt[215f.]{himmelmann1997}; \citealt[142f., 311]{lyons1999}; \citealt[140]{maho1998}; \citealt[18]{rust1965}\\
kxoe1243	& Khwe (Kxoe) 		& Khoe-Khwadi 			& \citealp[40--43, 79]{kilianhatz2008}\\
\midrule
\multicolumn{4}{c}{{\emph{Niger-Congo languages (8 languages/4 genera)}}}\\
\midrule
kiny1244	& Kinyarwanda 		& Bantu 	& \citealp[68f.]{vanderwal2022}\\
%Kirundi & Niger-Congo, Bantu & \\
%Lubukusu & Niger-Congo, Bantu & \\
gand1255	& Luganda 		& Bantu 	& \citealt[102]{ashtonetal1954}\\
nkor1241	& Nkore-Kiga 		& Bantu 	& (Q) \citealp[131]{taylor1985}\\
nzad1234	& Nzadi			& Bantu*	& \citealp[210, 279]{craneetal2011}\\
swah1253	& Swahili 		& Bantu 	& Vital Kazimoto (pers. comm.)\\
veng1238	& Babungo 		& Wide Grassfields & (Q) \citealp[197f.]{schaub1985}\\
% Dagara & Niger-Congo, Gur & \\
koro1298	& Koromfe 		& Koromfe 	& (Q) \citealp[242, 250f.]{rennison1997}\\
% Dan & Niger-Congo, Mande & \\
% Samogokan & Niger-Congo, Mande & \\
supy1237	& Supyire 		& Senufo 	& \citealp[207f.]{carlson1994}\\
\midrule
\multicolumn{4}{c}{{\emph{Papuan languages (26 languages/19 genera)}}}\\
\midrule
meny1245	& Menya		& Nuclear Angan		& \citealp[40, 46, 56f.]{whitehead2006}; \citealp[9f., 18f.]{whitehead2013}\\
fass1245	& Momu		& Baibai-Fas	 	& \citealp[169f., 242, 568]{honeyman2016}\\
imon1245	& Imonda 	& Border	 	& \citealp[44, 61f.]{seiler1985}\\
bilu1245	& Bilua 	& Bilua 		& \citealp[47--49, 76, 79, 84f., 87--89, 92f.]{obata2003}\\
lavu1241	& Lavukaleve 	& Lavukaleve 		& \citealp[171--173]{terrill2003}, \citealp[435, (27)]{terrill2004}\\
%Yimas & Ramu-Lower Sepik, Karawari & \citealp{foley1991}\\
mosk1236	& Moskona	& East Bird's Head* 	& \citealp[91, 222--224, 344]{gravelle2010}\\
mani1235	& Sougb		& East Bird's Head	& \citealp[200, 269f., 274]{reesink2002sougb}\\
hata1243	& Hatam		& Hatim-Mansim		& \citealp[195]{reesink1999}\\
maib1239	& Maybrat	& Maybrat		& \citealp[141, 158, 172, (281)]{dol2007}\\
urim1252	& Urim		& Urim			& \citealp[123, 125]{hemmilaeluoma1987}\\
savo1255	& Savosavo	& Savosavo*		& \citealp[147,155--159]{wegener2012}\\
mana1298	& Manambu 	& Sepik, Ndu 		& \citealp[197f., 508--513]{aikhenvald2008}\\
awtu1239	& Awtuw 	& Sepik, Ram 		& \citealp[120--124]{feldman1986}\\
alam1246	& Alamblak 	& Sepik, Sepik Hill 	& \citealp[90--92, 96f.]{bruce1984}\\
fore1270	& Fore 		& \gls{tng}, Fore-Gimi 		& \citealp[79f., 100f.]{scott1978}\\
huaa1250	& Hua 		& \gls{tng}, Siane-Yagaria 	& \citealp[226--232, 239f.]{haiman1980}\\
yaga1260	& Yagaria (\emph{Move}) & \gls{tng}, Siane-Yagaria & \citealp[17f., 166, 181]{renck1975}\\
amel1241	& Amele 	& \gls{tng}, Mabuso 		& (Q) \citealp[162, 201, 209f.]{roberts1987amele}\\
kobo1249	& Kobon 	& \gls{tng}, Kalam-Kobon 	& (Q) \citealp[107f., 157]{davies1989}\\
usan1239	& Usan 		& \gls{tng}, North Adelbert 	& \citealp[53f., 167, 190f., 353]{reesink1987}\\
adan1251	& Adang 	& \gls{tng}, Alor-Pantar 	& \citealp[261]{robinsonhaan2014}\\
kaer1234	& Kaera 	& \gls{tng}, Alor-Pantar* 	& \citealp[129]{klamer2014}\\
kama1365	& Kamang 	& \gls{tng}, Alor-Pantar 	& \citealp[313f.]{schapper2014}\\
sawi1256	& Sawila 	& \gls{tng}, Alor-Pantar* 	& \citealp[391]{kratochvil2014}\\ 
wers1238	& Wersing 	& \gls{tng}, Alor-Pantar 	& \citealp[472]{schapperhendery2014}\\
%Teiwa & \gls{tng}, \gls{tap}, Pantar & \citealp{klamer2010}\\
lamm1241	& Western Pantar & \gls{tng}, Alor-Pantar* 	& \citealp[53f.]{holton2014}\\
%wara1294	& Komnzo	& Yam, Morehead-Maro\*	& \citealp[552]{doehler2016}\\
\midrule
\newpage	% for layout reasons
\multicolumn{4}{c}{{\emph{Sino-Tibetan (2 languages/2 genera)}}}\\
\midrule
gesh1238	& East Geshiza	& Burmo-Qiangic* 	& \citealp[388, 400, 438, 480, 507, 646]{honkasalo2019}\\
mand1415	& Mandarin 	& Chinese 		& \citealt[297--299]{huangetal2009}; \citealp[sec. 7.3]{boskovichsieh2013} \\
\midrule
\multicolumn{4}{c}{{\emph{Uralic (2 languages/2 genera)}}}\\
\midrule
hung1274	& Hungarian 	& Ugric 		& (Q) \citealp[269]{keneseietal1998}; \citealp[559]{hoehn2015unagr}\\
finn1318	& Finnish 	& Finnic 		& (Q) \citealp[277]{sulkalakarjalainen1992} \\
\midrule
\multicolumn{4}{c}{{\emph{Creoles (3 languages/1 "genus")}}}\\
\midrule
ndyu1242	& Ndyuka 	& Creoles 		 	& (Q) \citealp[224, 460, 466f.]{huttarhuttar1994}\\
nige1257	& Nigerian Pidgin & Creoles 		 	& (Q) \citealp[178, 181]{faraclas1996}\\
mala1533	& Kristang 	& Creoles 			& \citealp[86]{baxter1988}\\
\midrule
\multicolumn{4}{c}{{\emph{Various (15 languages/15 genera)}}}\\
\midrule
mikm1235	& Mi’kmaq 	& Algonquian			& \citealp[188]{pacifiqueetal1990}\\
nucl1297	& Katu 		& Katuic			& \citealp[28]{costello1969}\\
wari1268	& Wari' 	& Chapacura-Wanham		& (Q) \citealp[303, 310]{everettkern1997}\\
chit1248	& Chitimacha	& Chitimacha			& \citealp[333]{swadesh1967}\\
lezg1247	& Lezgian 	& Lezgic			& \citealp[259]{haspelmath1993}\\
abkh1244	& Abkhaz 	& Northwest Caucasian 		& (Q) \citealp[157, 159]{hewitt1989}\\
basq1248	& Basque 	& Basque 			& (Q) \citealp[210]{saltarelli1988}; \citealp[122]{trask2003}; \citealp[482, 501f.]{derijk2008}; \citealp[67]{areta2009}; \citealp{artiagoitia2012DP}\\
clas1250	& Classical Nahuatl & Aztecan	 		& \citealp[192--194]{andrews1975}; \citealp[ch. 17.3]{andrews2003}\\
even1259	& Evenki 	& Tungusic 			& (Q) \citealp[197, 199]{nedjalkov1997}\\
hixk1239	& Hixkaryana 	& Cariban 			& (Q) \citealp[131]{derbyshire1979}\\
nucl1643	& Japanese 	& Japanese 			& (Q) \citealp[254, 261]{hinds1988}; \citealp[780]{noguchi1997}; \citealp[sec. 3.2]{furuya2008}; \citealp{inokuma2009}\\
kore1280	& Korean 	& Korean 			& (Q) \citealp[284; 292]{sohn1994}; \citealp[151--154]{choi2014phd}\\
kala1399	& Kalaallisut (W. Greenlandic) 	& Eskimo	& (Q) \citealp[110, 253, 256f.]{fortescue1984}\\
lakk1238	& Lakkia	& Kadai		 		& \citealp[137]{fan2019}\\
nucl1301	& Turkish 	& Turkic 			& (Q) \citealp[288, 297f.]{kornfilt1997}\\
\bottomrule
%\end{mpsupertabular}
%\end{table}
%\end{center}
\end{longtable}


%%%%%%%%%%%%%%%%%%%%%%%%%%%%%%%%%%%%%%%%%%%%%%%%%%%%%%%%%%%%%
%\clearpage
\section{Annotation scheme of csv (Supplementary Material S1)}

\begin{description}
\item[language] language name
\item[glottolog] glottocode following Glottolog \citep{glottolog4.5}
\item[sortclass] manual coding of phylogenetic information purely for more accessible sorting of data presentation 
\item[family] language family based on Glottolog \citep{glottolog4.5}
\item[genus.print] genus information for presentation purposes
\item[genus.WALS] genus following WALS \citep{wals} where available, otherwise manually supplied
\item[printname] language name used on world map
\item[Oc.printname] language name used on maps for Oceania
\item[print] set to $1$ if language printed on map
\item[lat] geographical coordinates based on WALS and glottolog, latitude
\item[lat] geographical coordinates based on WALS and glottolog, longitude
\item[hjust] horizontal adjustment for printing language name on world map
\item[vjust] vertical adjustment for printing language name on world map
\item[Oc.hjust] horizontal adjustment for printing language name on map for Oceania
\item[Oc.vjust] vertical adjustment for printing language name on map for Oceania
\item[area.glottolog] area information based on glottolog \citep{glottolog4.5}
\item[area.WALS] area information based on WALS \citep{wals}
\item[ref.NomPers] bibliographic information on \gls{persn}; (Q) indicates questionnaire-based Routledge grammars
\item[NomPers] value \textbf{y} iff information on \gls{persn} available; \textbf{n} if not
\item[APC] value \textbf{y} iff \glspl{apc} attested; \textbf{n} if not
\item[WO.WALS] verb-object order based on \citet{wals-83}
\item[WO] data from WO.WALS and manually supplied values for gaps
\item[adpos.manual] manually collected adposition-noun order data
\item[adpos.WALS] adposition-noun order based on \citet{wals-85}
\item[adpos] data from adpos.WALS and manually supplied values for gaps
\item[genitive.WALS] genitive-noun order based on \citet{wals-86}
\item[genitive] data from genitive.WALS and manually supplied values for gaps
\item[DemDir.WALS] demonstrative-noun order based on \citet{wals-88}
\item[DemDir.manual] manually collected demonstrative-noun order data
\item[DemDir] data from DemDir.WALS and manually supplied values for gaps
\item[genitive.WALS] genitive-noun order based on \citet{wals-86}
\item[genitive] data from genitive.WALS and manually supplied values for gaps
\item[ArtOrder] order of article and noun if available, \textbf{NA} if no articles
\item[Art.with.Nompers] value \textbf{y} iff articles attested in \glspl{apc}, \textbf{n} if not possible, \textbf{unclear} if unknown, \textbf{NA} if language without articles
\item[APC.dir] value \textbf{pre} iff prenominal \glspl{apc}, \textbf{post} for postnominal \glspl{apc}, \textbf{both} for ambidirectional \glspl{apc}, \textbf{apc} if no \glspl{apc} attested
\item[boundPers] value \textbf{pre} iff prenominal \glspl{bpc}, \textbf{post} for postnominal \glspl{bpc}, \textbf{n} if no bound \gls{persn}-marking
\item[boundPers.simultAPC] value \textbf{pre} if prenominal \gls{apc}-like marking attested simultaneously with bound \gls{persn}-marking, \textbf{post} if postnominal \gls{apc}-like marking attested simultaneously with bound \gls{persn}-marking, \textbf{NA} otherwise
\item[PPDC] value \textbf{y} iff \glspl{ppdc} attested; \textbf{y-nn} if attested \glspl{ppdc} include only demonstrative and \gls{persn}-marking, but no lexical material; \textbf{NA} if not attested
\item[PPDC.ref] bibliographic references for \glspl{ppdc}
\end{description}



%%%%%%%%%%%%%%%%%%%%%%%%%%%%%%%%%%%%%%%%%%%%%%%%%%%%%%%%%%%

\section{List of examples}

\gls{persn} markers are set in \textbf{bold} for ease of access.


\subsection{Afroasiatic languages}


\subsubsection{Hausa (haus1257), West Chadic}

\pex
\a \begingl
\gla \textbf{m\={u}} Háus\textgravemacron{a}w\={a}//
\glb we Hausa//
\glft `we Hausa'\\\citep[371]{newman2000}//
\endgl
\a \begingl
\gla \textbf{sh\={\textsci}} wannàn m\={a}làm\={\textsci}//
\glb he \Dem{}.1 teacher//
\glft `he (this) teacher'\\{\citep[after][371]{newman2000}}//
\endgl
\a
\begingl
\gla \textbf{m\={u}} m\textgravemacron{a}làman-nàn//
\glb we teacher-\Dem.\Prox{}//
\glft `we these teachers'\\\citep[155]{newman2000}//
\endgl
\xe

See \citet[63, 155, 370f.]{newman2000} and also \citet[330f.]{jaggar2001} for further examples.




\subsubsection{Mupun (mwag1236), West Chadic}
 
\ex
\begingl
\gla \textbf{war} manaja n\textschwa//
\glb 3\F{} manager \Def{}//
\glft `she, the manager'\\\citep[after][172, (154)]{frajzyngier1993}//
\endgl
\xe
 
\subsubsection{Gorwaa (goro1270), Southern Cushitic}

\ex
\begingl
\gla \textbf{atén} oo hhawató//
\glb Pro1\Pl{} \Anaph.\M{} men.\Lnk.\M{}//
\glft `we men'\\\citep[after][163, (2.205)]{harvey2018}//
\endgl
\xe

\subsubsection{Kambaata (kamb1316), East Cushitic}

\ex 
\begingl
\gla \textbf{na'óot} Kambáat-u//
\glb 1\Pl.\Nom{} Kambaata-\M.\Nom{}//
\glft `we Kambaata (people)' \\\citep[335, (1097)]{treis2008}//
\endgl
\xe

\subsubsection{Cairene Egyptian Colloquial Arabic (egyp1253), Semitic}

\ex
\begingl
\gla \textbf{\textglotstop{}intu} \textglotstop{}it[-]talamza ti\textcrh{}ibbu \textglotstop{}illi\textrevglotstop{}b//
\glb you.\Pl{} [\Def-]students 2\Pl{}.like playing//
\glft `You students like playing.'\\\citep[80, (533)]{garygamal1982}//
\endgl
\xe

See \citealp[78; 80]{garygamal1982}.

\subsubsection{Gulf Arabic (gulf1241), Semitic}


\ex
\begingl
\gla \textbf{iHna} T-Tullaab ma nigdar nigbal haadha l-qaraar//
\glb we \Def-students not 1\Pl.be.able 1\Pl.accept this \Def-decision//
\glft `We students cannot accept this decision'\\\citep[165, (845)]{holes1990}//
\endgl
\xe


See \citealp[162, 165]{holes1990}.

\subsubsection{Maltese (malt1254), Semitic}

\ex
\begingl
\gla {}[\textbf{Intom} il-\textcrh{}addiema] g\textcrh{}andkom ting\textcrh{}aqdu//
\glb you \Def-workers have.2\Pl{} unite.2\Pl{}//
\glft `You workmen should unite together.' \\{\citep[202, (915)]{borgazzopardialexander1997}}//
\endgl
\xe 


%%%%%%%%%%%%%%%
\subsection{Australian languages}

\subsubsection{Mangarrayi (mang1381), Mangarrayi}

\ex
\begingl
\gla \textbf{\ng{}j\d{l}a} malam-ga\d{l}a ga-\ng{}i\d{l}a-\d{n}i \d{n}a-wa\d{l}ayjñin-gan//
\glb 1\Pl.\Excl.\Nom{} man-\Pl.\Nom{} 3-1\Pl.\Excl-sit NLoc-shade//
\glft `We men are sitting in the shade.'\\\citep[103]{merlan1989}//
\endgl
\xe

See \citealp[103; 203]{merlan1989} for discussion.


\newpage
\subsubsection{Diyari (dier1241), Central Pama-Nyungan}


\pex 
\a \begingl
\gla \textbf{ngayani} waka-li thananha nhayi-yi//
\glb 1\Pl.\Excl.\Nom{} small-\Erg{} 3\Pl.\Acc{} see-\Prs{}//
\glft `We small (ones) watch them'\\\citep[102, (106)]{austin2013}//
\endgl
\a 
\begingl
\gla ngarda-nhi thana-li nhinha yakalka-yi yaru-ya wardayari-lha \textbf{yini} pinarru//
\glb then-\Loc{} 3\Pl.\Erg{} 3\Sg.\Nf.\Acc{} ask-\Prs{} like-\Dem.\Prox{} where-\Char{} 2\Sg.\Nom{} old.man.\Nom//
\glft `Then they asked him the following: ``Where are you from, old man?'' '\\\citep[119, (175)]{austin2013}//
\endgl
\a\begingl
\gla \textbf{nhani} mankarra thurara-yi thana-ngu-ya kinthala-nhi//
\glb 3\Sg.\F.\Nom{} girl.\Nom{} lie-\Prs{} 3\Pl-\Loc-\Prox{} dog-\Loc{}//
\glft `The girl is lying with those dogs'\\\citep[150, (311)]{austin2013}//
\endgl
\xe


For discussion and more examples see \citet[97f.]{austin1981} and \citep[100--103, 119, 150]{austin2013}.

\subsubsection{Warlpiri (warl1254), Western Pama-Nyungan}

\pex \a 
\begingl
\gla \ng{}arka \textbf{njanu\ng{}u} ka pu\d{l}a-mi//
\glb man 3 \Prs{} shout-\Npst{}//
\glft `The aforementioned man is shouting'\\\citep[316, (22)]{hale1973}//
\endgl
\a
\begingl
\gla yapa \textbf{\ng{}atju}//
\glb person 1\Sg{}//
\glft `I person'\\\citep[317]{hale1973}//
\endgl
\xe

See also \citealp[70]{reece1970} and especially \citealp[316f.]{hale1973} for further discussion.


\newpage

\subsubsection{Pitjantjatjara (pitj1243), Western Pama-Nyungan}

\pex
\a
\begingl
\gla Minyma \textbf{palu\underline{r}u} ngayu-nya nya-ngu//
\glb woman 3\Sg{}.\Nom{} 1\Sg{}.\Acc{} see-\Pst{}//
\glft `The woman saw me.' \\{\citep[31, (110)]{bowe1990}}//
\endgl
\a 
\begingl
\gla \textbf{Palu\underline{r}u} wati nyara wa\underline{r}a-ngku mutaka palya-nu//
\glb 3\Sg{}.\Nom{} man distant tall-\Erg{} car fix-\Pst{}//
\glft `The tall man over there (in contrast to the other one) fixed the car.' \\{\citep[34, (114)]{bowe1990}}//
\endgl
\xe

See \citet[49--51]{bowe1990} for discussion.

\subsubsection{Guugu Yimidhirr (gugu1255), Northern Pama-Nyungan}

\pex
\a
\begingl
\gla \textbf{Nyulu} nhayun waarigan gada-y waanggu=wunaarna-y.//
\glb 3\Sg.\Nom{} that.\Abs{} moon.\Abs{} come-\Pst{} sleep=lie+\Redup-\Pst//
\glft `[Then] the Moon came and lay down to sleep.'\\\citep[157, (423)]{haviland1979}//
\endgl
\a
\begingl
\gla Bidha \textbf{nyulu} biini.//
\glb child.\Abs{} 3\Sg.\Nom{} die.\Pst{}//
\glft `The child died.'\\\citep[157, (424a)]{haviland1979}//
\endgl
\xe

For general description suggesting that ``the norm arrangement for an NP that refers to a human'' involves a prenominal \gls{apc} see \citet[104]{haviland1979}.

\subsubsection{Kuku Yalanji (kuku1273), Northern Pama-Nyungan}

\pex
\a 
\begingl
\gla \textbf{nyulu} jalbu//
\glb 3\Sg{} woman//
\glft `\textbf{the} woman'\\{\citep[after][202; gloss extrapolated]{patz2002}}//
\endgl
\a
\begingl
\gla \textbf{Yurra} karrkay dunga-y bana mana!//
\glb 2\Pl{}.\Nom{}(\Sarg{}) child.\Abs{}(\Sarg{}) go-\Imp{} water.\Abs{}(\Obj{}) get.\Imp{}//
\glft `You children go and get water!' \\{\citep[after][203, (620)]{patz2002}}//
\endgl
\xe

\pex
\a 
\begingl
\gla Pastor \textbf{nyulu}\ldots//
\glb pastor 3\Sg{}//
\glft `Pastor, he\ldots'\\{\citep[after][202, (611); gloss extrapolated]{patz2002}}//
\endgl
\a
\begingl
\gla Ngayu babi wilbuman yindu ngamu \textbf{nganjin} dunga-ri-ny mayi baka-nka.//
\glb 1\Sg{}.\Nom{}(\Sarg{}) father's.mother.\Abs{}(\Sarg) old.woman.\Abs{}(\Sarg{}) other.\Abs{}(\Sarg{}) mother.\Abs{}(\Sarg{}) 1\Pl{}.\Excl{}.\Nom{}(\Sarg{}) go-\Pl{}-\Pst{} food.\Abs{}(\Obj{}) dig-\Purp//
\glft `I, grandmother, another old woman and mother, we went out to dig for food (yams).' \\{\citep[after][203, (618)]{patz2002}}//
\endgl
\xe

See \citet[120f., 202f.]{patz2002}.

\subsubsection{Kayardild (kaya1319), Tangkic}

\ex
\begingl
\gla \textbf{niya} jungarra dangkaa//
\glb he big man//
\glft `the big man'\\\citep[239]{evans1995}//
\endgl
\xe

See also \citet[239; 251]{evans1995} and \citet[141]{round2013}.


%%%%%%%%%%%%%%%%%%%%%%%%%%%%%%%%%%%%
\subsection{Austronesian languages}

\subsubsection{Malagasy (mala1537), Barito}

\ex
\begingl
\gla \textbf{izaho} vehivavy//
\glb 1\Sg.\Def{} woman//
\glft `I woman'\\\citep[411, (6a)]{paultravis2019}//
\endgl
\xe

For further discussion of some complex variation patterns see \citet{paultravis2019}.
 
 
 \newpage
\subsubsection{Indonesian (stin1234), Malayo-Sumbawan}
	
\ex \begingl
\gla \textbf{kami}, bangsa Indonesia//
\glb we people Indonesia//
\glft `we, the people of Indonesia'\\\citep[170]{sneddon1996}//
\endgl
\xe

\subsubsection{Papuan Malay (papu1250), Malayo-Sumbawan}

\ex
\begingl
\gla de blang, a, \textbf{om} \textbf{ko} \textbf{ini} tra liat\ldots//
\glb 3\Sg{} say ah! uncle 2\Sg{} \Dem.\Prox{} \Neg{} see//
\glft `he said, ``ah, \textbf{you uncle there} didn't see\ldots'' '\\\citep[353, (66)]{kluge2017}//
\endgl
\xe

See \citet[ch. 6.2]{kluge2017} for detailed discussion and more examples.

\subsubsection{Loniu (loni1238), Oceanic}
	

\pex
\a \begingl
\gla \textbf{s\textepsilon{}h} pihin s\textepsilon{}h \v{c}ani \textbf{uw\textepsilon{}h} kaman uw\textepsilon{}h w\textepsilon{}\v{c}\textepsilon{} ake//
\glb 3\Pl{} woman 3\Pl{} clear 1\Pl.\Excl{} man 1\Pl.\Excl{} cut.down tree//
\glft `The women clear, we men cut down the trees.'\\\citep[90, (6)]{hamel1994}//
\endgl
\a 
\begingl
\gla \textbf{iy} amat iy\textopeno{}//
\glb 3\Sg{} man \Dem//
\glft `this man'\\\citep[100, (89)]{hamel1994}//
\endgl
\xe

See \citet[sec. 4.2.1]{hamel1994} for discussion.

\subsubsection{Windesi Wamesa (wame1241), Oceanic}

\ex
\begingl
\gla sinitu=pa-\textbf{tata}//
\glb person=\Det{}-1\Pl{}.\Incl{}//
\glft `we people' \\{\citep[144, (3.46)]{gasser2014}}//
\endgl
\xe

\subsubsection{Maori (maor1246), Oceanic}

\ex 
\begingl
\gla E kaha rawa atu [\textbf{maatou} ngaa  kaiako naa] ki te pata\textperiodcentered{}patai//
\glb \Tam{} strong very away 1\Pl.\Excl{} the.\Pl{} teacher \Dem.2 to the \Redup{}\textperiodcentered{}ask//
\glft `We teachers ask a lot of questions.'\\\citep[373, (1673)]{bauer1993}//
\endgl
\xe

For discussion and more examples see \citealp[368, 373]{bauer1993} and \citealp[262f.]{bauer1997}.

\subsubsection{Tuvaluan (tuva1244), Oceanic}

\pex \a
\begingl
\gla \textbf{Au} ttino poto koo leva ne iloa nee au mea kolaa faatoaa iloa nee koe ttagata valea.//
\glb I the+person intelligent \Pfv{} {} {} know \Erg{} I thing those just know \Erg{} you the+man stupid//
\glft `I, an intelligent person, have long known what you, stupid man, are just discovering.'//
\endgl
\a\begingl
\gla \textbf{Taatou} tino Tuuvalu e see tau ki meakkai kolaa.//
\glb we.\Incl{} person Tuvalu \Npst{} \Neg{} befit to food those//
\glft `We Tuvaluans are not accustomed to that [type of] food.'\\\citep[393, (2018/2019)]{besnier2000}//
\endgl
\xe

See \citealp[392f.]{besnier2000}.

\subsubsection{Kwaio (kwai1243), Oceanic}

\pex
\a \begingl
\gla \textbf{'a-gauru-a} ta'a i 'Ai'eda//
\glb  \Fpron.3\Tri-? people \Loc{} 'Ai'eda//
\glft `those 'Ai'eda people'//
\endgl
\a
\begingl
\gla fa-\textbf{meru-a} ta'a geni//
\glb for-\Spron.1\Tri.\Excl-? people female//
\glft `for us women' \\\citep[after][104]{keesing1985}//
\endgl
\xe

The status of the \emph{-a} intervening between pronoun and noun is not clear, see \citet[25]{hoehn2020ThirdGap} for speculation that this might be a reduced article.

\subsubsection{Arop-Lokep (arop1243), Oceanic}

\ex
\begingl
\gla \textbf{am} garup ke Bok//
\glb 1\Excl.\Pl{} female.one of Bok//
\glft `we women of Bok village'\\\citep[255]{djernes2002}//
\endgl
\xe


\subsubsection{Cheke Holo (chek1238), Oceanic}

\pex 
\a \begingl
\gla \textbf{Tahati} naikno \={g}re e kmana pui~puhi=da//
\glb 1\Pl.\Incl{} people \Dem.\Prox.\Pl{} \Emph{} lot.of \Dur~way=1\Pl.\Poss{}//
\glft `We people have had many problems.'\\\citep[165, (591)]{boswell2018}//
\endgl
\a \begingl
\gla \textbf{Gotilo} Honiara fa-le~lehe egu//
\glb 2\Pl{} Honiara \Caus-\Dur~die like.that//
\glft `You people of Honiara kill [people], like that.'\\\citep[99, (301)]{boswell2018}//
\endgl
\xe

%See \citealp[165]{boswell2018}.

\subsubsection{Hoava (hoav1238), Oceanic}


\ex \begingl
\gla Maki lavati sa pa Solomone, gi ta-\underline{n}ani \textbf{gita} nikana hupa.//
\glb \Neg{}.\Imp{} big 3\Sg{} \Loc{} Solomons and \Pass-eat 1\Pl.\Incl{} man black//
\glft `Let it [a monkey] not grow big in the Solomons, and we black men be bitten.'\\\citep[after][48, (72a)]{davis2003}//
\endgl
\xe

See \citealp{palmer2017} for detailed discussion.

\subsubsection{Kokota (koko1269), Oceanic}

\pex
\a 
\begingl
\gla \textbf{gai} nakoni zuzufra//
\glb we.\Excl{} person black//
\glft `we black people'\\\citep[95, (3.70a)]{palmer2008}//
\endgl
\a 
\begingl
\gla ka \textbf{gai} ira nakoni zuzufra tana nogoi naito tahi ke a\={g}e=u=ni=u//
\glb \Loc{} we.\Incl{} the.\Pl{} person black then \Voc{} devil sea \Pfv{} go=be.thus=3\Sg.\Obj=\Cnt//
\glft `With us black people, then, man!, `sea devil' is what it's called.' \\\citep[123, (4.1b)]{palmer2008}//
\endgl
\xe

See \citet[68, 95, 116, 119, 123, 131, 137, 163, 242, 300, 305, 327, 399, 414]{palmer2008} for various examples.

%%%%%%%%%%%%%%%%%%%%%%%%%%%%%%%%%%
\subsection{Dravidian languages}


\subsubsection{Kannada (nucl1305), Dravidian}

\ex \begingl
\gla \textbf{na:vu} bra:hmaNaru vidyeyannu mareyaba:radu.//
\glb we Brahmin.\Pl{} education.\Acc{} forget.\Inf.\Proh//
\glft `We Brahmins shouldn't forget education.'\\\citep[209, (755)]{sridhar1990}//
\endgl
\xe

See \citet[205, 208f.]{sridhar1990}.

\subsubsection{Malayalam (mala1464), Dravidian}

\ex \begingl
\gla \textbf{\textltailn{}a\ng\ng{}a\textrtaill{}} intyakkaar adhikavum adhikavum sasyabhuukkuka\textrtaill{} aa\textrtailn{}\textschwa{}//
\glb we.\Excl{} Indians mostly vegetarian.\Pl{} be.\Prs{}//
\glft `We Indians are mostly vegetarians.'\\\citep[262, (1340)]{asherkumari1997}//
\endgl
\xe 

See  \citealp[262f.]{asherkumari1997} for more examples.

\subsubsection{Tamil (tami1289), Dravidian}

\ex \begingl
\gla \textbf{naa\ng{}ka} i\ng{}lii\c{s}kaara\ng{}kellaam kaaramaana vastu caappi\textrtailt{}aratille//
\glb we English.person.\Pl.all spicy thing eat.\Prs.\Nom{}.\Neg//
\glft `We English do not eat hot things.'\\\citep[146]{asher1985}//
\endgl
\xe 

See \citet[142, 146]{asher1985}.

%%%%%%%%%%%%%%%%%%%%%%%%%%%%%%%%%%%%
\subsection{Indo-European languages}


\subsubsection{Welsh (wels1247), Celtic}

\ex \begingl
\gla \textbf{ni} fyfyrwyr//
\glb we \Softmut.student.\Pl{}//
\glft 'we students'\\(David Willis, personal communication)//
\endgl
\xe

\subsubsection{Danish (dani1285), Germanic}

\ex \begingl
\gla lad \textbf{os} voksne snakke i fred//
\glb let.\Imp{} 1.\Pl.\Acc{} adults talk.\Inf{} in peace//
\glft `Let us adults talk in peace.’\\\citep[29, (32b)]{schroeter2021}//
\endgl
\xe

See \citet{schroeter2021} for detailed discussion and \citet{johannessen2008} for psychologically distant demonstratives, corresponding to adnominal third person singular pronouns. 

\subsubsection{Icelandic (icel1247), Germanic}

\ex
\begingl
\gla við/ þið Íslendigar//
\glb we you.\Pl{} Icelanders//
\glft `we/ you Icelanders'\\\citep[11, (16a)]{sigurdssonwood2020}//
\endgl
\xe

See \citet[11f.]{sigurdssonwood2020} for further examples and \citet{johannessen2008} for psychologically distant demonstratives, corresponding to adnominal third person singular pronouns. 

\subsubsection{Norwegian (norw1258), Germanic}

\ex 
\begingl
\gla til \textbf{oss} (to) (gaml-e) professor-a-ne//
\glb to us two old-\Def{} professor-\Pl-\Def{}// 
\glft `to us (two) (old) professors' \\\citep[129, (4.36)]{julien2005}//
\endgl
\xe

See also \citet{johannessen2008} for psychologically distant demonstratives, corresponding to adnominal third person singular pronouns. 

\subsubsection{Swedish (swed1254), Germanic}

\ex 
\begingl
\gla \textbf{vi} student-er//
\glb we student-\Pl{}//
\glft `we students'\\\citep[128, fn. 18, (ia)]{julien2005}//
\endgl
\xe

See also \citet{johannessen2008} for psychologically distant demonstratives, corresponding to adnominal third person singular pronouns. 

\subsubsection{Dutch (dutc1256), Germanic}

\ex
\begingl
\gla \textbf{Wij}/ \textbf{jullie} taalkundigen denken te veel na.//
\glb we you.\Pl{} linguists think too much \Prtcl{}//
\glft `We/you linguists think too much.'\\\citep[52, (23a)]{corver2008vocative}//
\endgl
\xe 


\subsubsection{English (stan1293), Germanic}

\ex \textbf{you} linguists\xe

See among others: \citet{postal1969, delormedougherty1972, sommerstein1972, pesetsky1978, keizer2016}.

\subsubsection{German (stan1295), Germanic}

\ex
\begingl
\gla Wenn noch nicht einmal [\textbf{du} Linguist] die + neue Rechtschreibung beherrschst\ldots//
\glb if still \Neg{} \Prtcl{} you.\Sg{} linguist \Det.\Acc.\Sg{} new orthography command.2\Sg{}//
\glft `If not even you linguist have a command of the new orthography rules\ldots' \\{\citep[after][100, (36)]{rauh2004}}//
\endgl
\xe

See among others: \citet[ch. 6]{lawrenz1993}, \citet{rauh2003, rauh2004} and \citet{roehrs2005}.

\subsubsection{Standard Modern Greek (mode1248), Greek}

\ex 
\begingl
\gla \textbf{Emis} i ghlosoloji piname.//
\glb we \Det.\Pl{} linguists are.hungry.1\Pl{}//
\glft `We linguists are starving/hungry.'\\\citep[after][114, (12c)]{lekakouszendroi2012}//
\endgl
\xe 

See \citet{stavrou1995} as well as \citet[chs. 1/2]{choi2014phd} and \citet[sec. 5]{hoehn2015unagr}.

\subsubsection{Calabrian Greek/Greko (aspr1238), Greek}

\ex
\begingl
\gla \textbf{Emì} ta pedìa den pìnnome kafè.//
\glb we \Det.\N.\Pl{} children.\N{} \Neg{} drink.1\Pl{} coffee//
\glft `We children don't drink coffee.'\\\citep[after][274, (17)]{hoehnetalICGLcalabria}//
\endgl
\xe

See \citet{hoehnetalICGLcalabria} for further discussion.

\subsubsection{Kashmiri (kash1277), Indic}

\ex 
\begingl
\gla \textbf{\textschwa{}s'} k\textschwa{}:shir' chi k\textschwa{}:r'gar.//
\glb we Kashmiris are artisans//
\glft `We Kashmiris are artisans.'\\\citep[200, (8)]{walikoul1997}//
\endgl
\xe

See \citet[200]{walikoul1997}.

\subsubsection{Marathi (mara1378), Indic}

\ex \begingl
\gla \textbf{\={a}mh\={i}} b\={a}yak\={a} nehm\={i}ts \={a}ply\={a} kutumb\={ã}n s\={a}\d{t}h\={i} khapto//
\glb we women always-\Emph{} \Refl{} families for {work.hard-1\Pl}//
\glft `We women always work hard for the sake of our families.'\\\citep[386, (1141)]{pandharipande1997}//
\endgl
\xe

See \citet[381, 386]{pandharipande1997} for more examples.


\subsubsection{Punjabi (panj1256), Indic}

\ex
\begingl
\gla \textbf{as\={ii}} pañjaabii garam mijaaj de ãã.//
\glb we Punjabi hot nature \Gen.\M.\Pl{} are//
\glft `We Punjabis are hot blooded/tempered.'\\\citep[228, (703)]{bhatia1993}//
\endgl
\xe

See \citet[228]{bhatia1993}.

\subsubsection{Persian (west2369), Iranian}

\ex 
\begingl
\gla \textbf{ma} irani-a//
\glb we Iranian-\Pl//
\glft `we Iranians'\\\citep[212, (356)]{mahootian1997}//
\endgl
\xe

See \citet[209, 212]{mahootian1997}.

\subsubsection{Aromanian (arom1237), Romance}

\ex \begingl
\gla \textbf{noi} pikurar-li adrem pini.//
\glb we shepherd-\Def.\Pl{} baled bread//
\glft `We shepherds baked bread.'\\(elicited)//
\endgl
\xe

See also \citet[560]{hoehn2015unagr}. 

\subsubsection{Romanian (roma1327), Romance}

\ex \begingl
\gla \textbf{Voi} avoca\cb{t}ii v\u{a} ap\u{a}ra\cb{t}i clien\cb{t}ii.//
\glb you.\Pl{} lawyers-\Def{} \Cl.2\Pl{} defend cuzstomers-\Def{}//
\glft `You lawyers defend your clients.'\\\citep[10, (20a)]{cornilescunicolae2014}//
\endgl
\xe

For further examples see \citet[255, 258]{mallinson1986} and \citet[6, 10, 20f.]{cornilescunicolae2014}.

\subsubsection{Catalan (stan1289), Romance}

\ex \begingl
\gla \textbf{nosaltres} els bombers//
\glb we \Det.\Pl{} firemen//
\glft `we firemen'\\\citep[290]{hualde1992catalan}//
\endgl
\xe 

See also \citet[287, 290]{hualde1992catalan} and \citet[560]{hoehn2015unagr}.

\subsubsection{Galician (gali1258), Romance}

\ex
\begingl
\gla \textbf{nos} os estudantes//
\glb we \Det.\Pl{} students//
\glft `we students'\\\citep[560, (36a)]{hoehn2015unagr}//
\endgl
\xe

See \citealp[560]{hoehn2015unagr}, also \citet[152, 301]{alvarezetal1986}.

\subsubsection{European Portuguese (port1283), Romance}

\ex 
\begingl
\gla \textbf{Nós} portugueses bebemos bom café.//
\glb we Portuguese drink.1\Pl{} good coffee//
\glft `We Portuguese drink good coffee.'\\\citep[555]{hoehn2015unagr}//
\endgl
\xe

\subsubsection{Spanish (stan1288), Romance}

\ex \begingl
\gla \textbf{Nosotros} los lingüistas somos listos.//
\glb we \Det.\Pl{} linguists be.1\Pl{} smart//
\glft `We linguists are smart.'\\\citep[210, (38a)]{choi2014phd}//
\endgl
\xe

See \citet[210f.]{choi2014phd} and \citet[560]{hoehn2015unagr} as well as \citet[145]{debruyne1995}.

\newpage

\subsubsection{Italian (ital1282), Romance}

\ex \begingl
\gla \textbf{noi}/ \textbf{voi} linguisti//
\glb we you.\Pl{} linguists//
\glft `we/you linguists'\\\citep[202, (21a)]{cardinaletti1994}//
\endgl
\xe

See \citet[202f.]{cardinaletti1994}, \citet[559]{hoehn2015unagr} and also \citet{hoehnetal2016CalabrUnagr,hoehnetalICGLcalabria}.

\subsubsection{Northern Calabrese (nort2612), Romance}

\ex \begingl
\gla \textbf{Nua} i quatrarə iucamə i cartə.//
\glb we \Det.\Pl{} children play.1\Pl{} \Det.\Pl{} cards//
\glft `We children play cards.'\\\citep[276, (23b)]{hoehnetalICGLcalabria}//
\endgl
\xe

See similarly \citet[142]{hoehnetal2016CalabrUnagr}.

\subsubsection{Southern Calabrese (sout2616), Romance}

\ex \begingl
\gla \textbf{Nui} i figghioli iocamu e carti.//
\glb we \Det.\Pl{} children play.1\Pl{} \Det.\Pl{} cards//
\glft `We children play cards.'\\\citep[276, (23c)]{hoehnetalICGLcalabria}//
\endgl
\xe

See similarly \citet[142]{hoehnetal2016CalabrUnagr}

\subsubsection{Russian (russ1263), Slavic}

\ex \begingl
\gla \textbf{my} leningradcy//
\glb we Leningrader.\Pl//
\glft `we Leningraders'\\\citep[352, (4c)]{pesetsky1978}//
\endgl
\xe

\newpage

\subsubsection{Bulgarian (bulg1262), Slavic}

\ex \begingl
\gla \textbf{nie} student-i-te//
\glb we students-\Pl-\Def//
\glft `we students'\\\citep[560]{hoehn2015unagr}//
\endgl
\xe

See also \citet{osenova2003} and \citet{norman2001} for discussion on unagreement phenomenon in Bulgarian.


\subsubsection{Pomak (poma1238), Slavic}

\ex
\begingl
\gla \textbf{'nami} Po'matsem-se no na p\textturnv{}'maga 'nikutri//
\glb we.\Dat{} Pomak.\Dat.\Pl{}-\Det.\Prox{} 1\Pl.\Acc{} \Neg{} help.3\Sg{} nobody//
\glft `Nobody helps us Pomaks.'\\ \citep[582]{papadimitriou2008}//
\endgl
\xe 

See also \citet[269--272]{hoehnPhD} for some related discussion.

\subsubsection{Polish (poli1260), Slavic}

\ex \begingl
\gla \textbf{My} lingwiści lubimy formalizację.//
\glb we linguists like.1\Pl{} formalisation//
\glft `We linguists like formalisation.'\\\citep[161f., (3b)]{rutkowski2002}//
\endgl
\xe


%%%%%%%%%%%%%%%%%%%%%%%%%%%%%%%%%
\subsection{Khoe-Kwadi languages}

\subsubsection{Khoekhoe/Nama (nama1264), Khoe-Kwadi}


\ex \begingl
\gla \textbf{sa} k\textbottomtiebar{h}oe-\textbf{ta} ké n\~{i} ra \textdoublepipe{}'o.//
\glb \Art{}.\Addr{} person-1\Pl{}.\Incl{}.\Common{} \Top{}? \Compel{} \Prog{} die//
\glft `We humans have to die.' \\{\citep[after][133, (27b)]{boehm1985}}//
\endgl
\xe

Thanks to Menán du Plessis for Khoekhoe glossing. Khoekhoe glosses for \gls{persn}-expressions are my interpretation of \citet{haacke1977}.
See \citet[133--145]{boehm1985} and \citet[140]{maho1998} and particularly \citet{haacke1976, haacke1977, haacke2013namamorph} for further discussion. 

\subsubsection{Khwe/Kxoe (kxoe1243), Khoe-Kwadi}

\ex
\begingl
\gla {Hè é} \textbf{tó} Khwé-\textbf{tò} dì gó\'{\textepsilon} à tó ò + \textbf{\textdoublepipe{}é} Qúva-\textbf{\textdoublepipe{}è} \textdoublebarpipe{}xà-á-tè à.//
\glb \Dem{} 2\Pl{}.\Common{} Khwe-2\Pl{}.\Common{} \Poss{} cattle \Obj{} 2\Pl{}.\Common{} \Poss{} 1\Pl{}.\M{} White-1\Pl{}.\M{} give-I-\Prs{} \Obj{}//
\glft `Here are yours, the Khwe's cows that we, the Whites, give you.' \\{(\citealp[41, (1)]{kilianhatz2008} quoting \citealp[514f.]{koehler1989})}//
\endgl
\xe

See \citet[ch. 3.1.2]{kilianhatz2008} for discussion and \citet[79]{kilianhatz2008} for a further example.

%%%%%%%%%%%%%%%%%%%%%%%%%%%%%%%
\subsection{Niger-Congo languages}


\subsubsection{Kinyarwanda (kiny1244), Bantu}


\ex \begingl
\gla N-ubah-a \textbf{wow} mu-kire.//
\glb 1\Sg.\Subj-respect-\Fv{} 2\Sg{} \Ncl{}1-rich.man//
\glft `I respect you rich man.' \\\citep[69, (73a)]{vanderwal2022}//
\endgl
\xe

See \citet[67--69]{vanderwal2022} for discussion.

\subsubsection{Luganda (gand1255), Bantu}

\ex \begingl
\gla \textbf{Ffe} abantu abaavu ffe tubonaabona.//
\glb we people poor 1\Pl{} suffer//
\glft `We poor people suffer.'\\(elicited)//
\endgl
\xe 

See also \citet[102]{ashtonetal1954} for further (unglossed) examples.

\newpage

\subsubsection{Nkore-Kiga (nkor1241), Bantu}

\ex 
\begingl
\gla \textbf{itwe} abanyankore ni-tu-hinga ebinyoobwa//
\glb we Banyankole \Prs.\Ipfv-1\Pl-cultivate groundnuts//
\glft `We Banyankole grow groundnuts.'\\\citep[after][131, (368)]{taylor1985}//
\endgl
\xe 


\subsubsection{Nzadi (nzad1234), Bantu}


\ex \begingl
\gla nt\^{\textschwa}m. y\textepsilon{} kó luzí\ng{} \textdownstep{}é \textbf{b\"{i}} andzéé b\v{i} k yéè ntswé ninyá b\textopeno{}. atá{[\ldots]}//
\glb taste and \Loc{} life of us Nzadi.people we \Neg.\Prs{} sell fish that \Neg{} even//
\glft `\ldots tasty. In the tradition of us Nzadi, we don't sell that fish. Even\ldots' \\\citep[279, (10)]{craneetal2011}//
\endgl
\xe 

See \citet[210, 279]{craneetal2011}.

\subsubsection{Swahili (swah1253), Bantu}


\ex 
\begingl
\gla \textbf{Nyinyi} wa-nafunzi m-me-cheka.//
\glb you.\Pl{} \Ncl{}2-student 2\Pl{}-\Pst-laugh//
\glft `You students laughed.'\\(elicited)//
\endgl
\xe

Concerning unagreement \citep{hurtado1985, ackemaneeleman2012unagr} in Swahili see \citet[546]{hoehn2015unagr} for a short note.

\subsubsection{Babungo (veng1238), Wide Grassfields}

\ex
\begingl
\gla  \textbf{yìa} v\'{\textbari}\textbari{} ndâa g\'{\textschwa} ntó'//
\glb we.\Excl{} people smithy go-\Prs{} palace//
\glft `We, the blacksmiths, go to the palace.'\\\citep[197, (134a)]{schaub1985}//
\endgl
\xe

See \citet[197f.]{schaub1985} for discussion.

\subsubsection{Koromfe (koro1298), Koromfe}

\ex 
\begingl
\gla \textbf{\textupsilon{}k\textopeno{}} (a) koromb\textturnv{}//
\glb we \Art{} Koromba//
\glft `we Koromba'\\\citep[251, (585)]{rennison1997})//
\endgl
\xe 

The article may be dropped in fast speech, for more discussion see  \citet[242, 250f.]{rennison1997}.

\subsubsection{Supyire (supy1237), Senufo}

\ex \begingl
\gla \textbf{wùu} shìin taanré//
\glb we person.\Ncl{}1.\Pl{} three//
\glft `we three'\\\citep[after][208, (45a)]{carlson1994}//
\endgl
\xe

See \citealp[207f.]{carlson1994}.


%%%%%%%%%%%%%%%%%%%%%%
\subsection{``Papuan'' languages (non-Austronesian languages of Oceania)}


\subsubsection{Menya (meny1245), Nuclear Angan}

\pex \label{ex:menyabpc}
\a \begingl
\gla Nyi tä=\ng{}ga=\ng{}i Matiu i=qu=k=\textbf{i} kuk\ng{}uä + hn=i yat\ng{}qä k-i-m=\ng{}qä=i.//
\glb 1\Sg{} this=\textsc{time}=\Gvn{} Matthew \Dem=\M={2\Sg}=\Obj{} talk \Indf=\F{} ask 2\Sg-do-1\Sg/\Irr=\textsc{goal}=\Ind{}//
\glft `I'm now going to ask you Matthew something.'//
\endgl
\a\begingl
\gla \textbf{Ne} ämaqä qokä i=qu=\textbf{ne} yiämisa\ng{}ä huiyi=nä qw ä-n-k-qäqu=i.//
\glb 1\Pl{} person man \Dem=\M={1\Pl} food other=\Foc{} \Cert{} \Ass-eat-\Pst/\Pfv{}-1\Pl/\Dso=\Ind{}//
\glft `Then we men ate some other food.'\\
\citep[30, (58/59)]{whitehead2006}//
\endgl
\xe

Possible person marking on an indefinite expression:

\ex \begingl
\gla Hn=qu=\textbf{ki} quwä ä-ma-t-qä=i//
\glb \Indf{}=\M=\textbf{2\Sg} steal \Ass-get-2\Sg/\Irr-\Generic=\Def{}//
\glft `Should one of you steal something\ldots'\\\citep[9, (18)]{whitehead2013}//
\endgl
\xe

See \citet[40, 46, 56f.]{whitehead2006} and \citet[9f., 18f.]{whitehead2013} for further examples and discussion.

\subsubsection{Momu (fass1245), Baibai-Fas}

\ex 
\begingl
\gla \textbf{Yery} mu, baso nenwu wu-ta-r-u.//
\glb 1\Pl{} women child belly \Inan:be.at-\Stvzr-1\Pl\Sg{}-\Nmlz//
\glft `For we women, we get pregnant (lit. children are in our bellies).'\\\citep[568, (4)]{honeyman2016}//
\endgl
\xe

See \citet[169f., 242, 568]{honeyman2016}.

\subsubsection{Imonda (imon1245), Border}


\ex
\begingl
\gla \textbf{ka} sebuhe t\~{o}g\~{o} fi-li-t//
\glb 1 devil thus do-\Emph-\Cf//
\glft `We devils should have done it like that.'\\\citep[61, (8)]{seiler1985}//
\endgl
\xe 

See \citet[44, 61f.]{seiler1985}. The possibility for postnominal \glspl{apc} is mentioned, but unfortunately no example is provided.

\subsubsection{Bilua (bilu1245), Bilua}

\pex
\a
\begingl 
\gla \textbf{enge}=a Solomoni=a=ma maba poso=\textbf{ngela}//
\glb 1\Pl{}.\Excl{}=\Lig{} Solomon=\Lig=3\Sg{}.\F{} person \Pl{}.\M{}=1\Pl{}.\Excl{}//
\glft `we, Solomon people' \\{\citep[85, (7.35)]{obata2003}}//
\endgl
\a 
\begingl
\gla \ldots{} lai=za=\textbf{mu}=\textbf{mela} inio me.//
\glb {} where=\Lig=3\Pl{}=2\Pl{}{} \Foc{}.\Nf{} 2\Pl{}//
\glft `\ldots you are people from where?' \\\citep[88, (7.49)]{obata2003}//
\endgl
\xe

See \citet[47--49, 76, 79, 84f., 87--89, 92f.]{obata2003} for a range of further examples.

\subsubsection{Lavukaleve (lavu1241), Lavukaleve}

\ex 
\begingl
\gla aka {malav} \textbf{e} roa-ru kiu-la-m.//
\glb then people 1\Pl{}.\Excl{} one.\Sg{}\M{}-none die-\Neg{}-\Sg{}\M{}//
\glft `And we, the people [lit: the people we] didn't die. [i.e. None of us people died.]'\\{\citep[after][171, (197)]{terrill2003}}//
\endgl
\xe

See \citet[171--173]{terrill2003} and \citet[435, (27)]{terrill2004}.

\subsubsection{Moskona (mosk1236), East Bird's Head}

\pex
\a
\begingl
\gla \textbf{mi}-osnok mi-en-ah-miy,  mi-en-ot jig miyes + mi-er tofi.//
\glb 1\Pl{}-person 1\Pl-\Dur{}-bathe 1\Pl-\Dur{}-stand \Loc{} clothes 1\Pl{}-wear hat//
\glft `we people bathe, wear clothes, wear hats\ldots' (stand in clothes = wear clothes)\\
\citep[344, (2)]{gravelle2010}//
\endgl
\a
\begingl
\gla \textbf{Eri} \textbf{i}-ejen(a) i-odog jig\ldots//
\glb they.\Pl{} 3\Pl-woman 3\Pl-pregnant \Loc{}//
\glft `(If/when) they the women are pregnant\ldots' \\
\citep[91, (38)]{gravelle2010}//
\endgl
\a 
\begingl
\gla \textbf{eri} Mosmir//
\glb they Maybrat//
\glft `Maybrat people/tribe'\\
\citep[224, (43b)]{gravelle2010}//
\endgl
\xe

See \citet[91, 222--224, 344]{gravelle2010}.
 
\subsubsection{Sougb (mani1235), East Bird's Head}

\ex 
\begingl
\gla \textbf{Emen} Sougb/ emen meijouhw dangga.//
\glb we.\Excl{} Sougb we.\Excl{} custom like.that//
\glft `We Sougb, that's our custom.' \\\citep[274, (40)]{reesink2002sougb}//
\endgl
\xe 

See \citet[200, 269f., 274]{reesink2002sougb}.

\subsubsection{Hatam (hata1243), Hatim-Mansim}

\ex \begingl
\gla \textbf{Nye}-ni sop Tinam hwop yok nya ci gi-mbres//
\glb we-this woman Tinam girl put \Pl{} chase-away \Nmlz{}-wide//
\glft `We Tinam women are not easy to take.'\\\citep[195, (72)]{reesink1999}//
\endgl
\xe 


\subsubsection{Maybrat (maib1239), Maybrat}

\ex
\begingl
\gla fnia \textbf{anu} p-no po re-t-o fawen fe//
\glb woman 2\Pl/1\Pl.\Incl{} 1\Pl-do thing location.\Spec-\Prox-\Unmark{} long.time \Neg{}//
\glft `We women (inc), we haven't done this thing for a long time.'\\\citep[158, (99)]{dol2007}//
\endgl
\xe

See \citet[64, fn. 5]{dol2007} concerning the 1\Pl.\Incl{} use of the pronoun \emph{anu}.
See \citet[141, 158, 172, (281)]{dol2007} for examples.

\subsubsection{Urim (urim1252), Urim}

\ex \begingl
\gla \textbf{men} melnum watipmen//
\glb 1\Pl.\Excl{} person many//
\glft `we(excl.) many people'\\
\citep[123]{hemmilaeluoma1987}//
\endgl
\xe 

\pex
\a
\begingl
\gla \textbf{tu} melnum//
\glb 3\Pl{} person//
\glft `the people'//
\endgl
\a \begingl
\gla \textbf{kupm} Melming la nak-etn por ti//
\glb 1\Sg{} Melming say tell.\Real{}-2\Sg.\Obj{} story this//
\glft `I Melming told you this story.'\\
\citep[125]{hemmilaeluoma1987}//
\endgl
\xe

See \citet[123, 125]{hemmilaeluoma1987}.

\subsubsection{Savosavo (savo1255), Savosavo}

\ex \begingl
\gla {}[[No \textbf{mapa=gha}]\unt{NP} [\textbf{ave}]\unt{NP}]\unt{NP}=na kula ata no-va nito=la.//
\glb 2\Sg.[\Gen] person=\Pl{} 1\Pl.\Excl=\Nom{} seawards here 2\Sg-\Gen.\M{} eye=\Loc.\M{}//
\glft `[Addressing the volcano:] We, your people, (are) here seawards at your eye.'\\\citep[155, (285)]{wegener2012}//
\endgl
\xe

See \citet[147,155--159]{wegener2012}.

\subsubsection{Manambu (mana1298), Sepik, Ndu}

\ex \begingl
\gla \textbf{wun} ñ\textschwa{}n-a ñam\textschwa{}s//
\glb I you.\F-\Lnk+\F.\Sg{} younger.sibling//
\glft `me, your younger sister'\\\citep[197]{aikhenvald2008}//
\endgl
\xe

See \citet[197f., 508--513]{aikhenvald2008} for further examples and discussion.

\subsubsection{Awtuw (awtu1239), Sepik, Ram}


\pex \a
\begingl
\gla \textbf{rom} Meley-y\ae{}nim//
\glb 3\Pl{} Meley-\Generic{}//
\glft `the people from Meley' \\\citep[122, (15a)]{feldman1986}//
\endgl
\a 
\begingl
\gla \textbf{rey}/ \textbf{tey} tale//
\glb 3\M.\Sg{}/ 3\F.\Sg{} woman//
\glft `the woman'\\\citep[123, (21a)]{feldman1986}//
\endgl
\xe

See \citet[120--124]{feldman1986} for discussion.

\newpage

\subsubsection{Alamblak (alam1246), Sepik, Sepik Hill}

\pex %\citet[96]{bruce1984} \trailingcitation{[Alamblak]}
\vspace{-1.832\baselineskip}
\begin{multicols}{3}
\a
\begingl 
\gla yima-\textbf{m}//
\glb person-3\Pl{}//
\glft `people'\\\citep[96]{bruce1984}//
\endgl
\a
\begingl
\gla yima-\textbf{k\"{e}}//
\glb person-2\Pl{}//
\glft `you people'//
\endgl
\a
\begingl
\gla yima-\textbf{n\"em}//
\glb person-1\Pl{}//
\glft `we people'//
\endgl
\end{multicols}
\xe

See \citet[90--92, 96f.]{bruce1984}.

\subsubsection{Fore (fore1270), \gls{tng}, Fore-Gimi}


\ex
\begingl
%\gla aogi yagaraná: kanauwe.//
\gla aogi yagara:'-\textbf{na:} kana-u-e//
\glb good man-1\Sg{} come-1\Sg-\Ind{}//
\glft `I, the good man, come.'\\{\citep[after][80]{scott1978}}//
\endgl
\xe

See \citet[79f., 100f.]{scott1978}.

\subsubsection{Hua (huaa1250), \gls{tng}, Siane-Yagaria}

\pex 
\a
\emph{Forapi' + da} $\rightarrow$ /\emph{forapi \textbf{da}}/ `I, Forapi'
\a
\emph{Forapi' + Ka} $\rightarrow$ /\emph{forapi\textbf{ga}}/ `You, Forapi'
\a
\emph{nono' + 'Kama' + da} $\rightarrow$ /\emph{nonokama \textbf{da}}/ `I your maternal uncle'\\{\citep[226]{haiman1980}}
\xe

\pex Person marked genitive forms{\citep[after][240]{haiman1980}}
\a vimata \newline`of us men'
\a ademata \newline `of us women'
\a vi'ita \newline `of you men'
\a adita \newline `of you women'
\xe

See \citet[226--232, 239f.]{haiman1980}.

\subsubsection{Yagaria (yaga1260), \gls{tng}, Siane-Yagaria}

\ex 
\begingl
\gla dagaea ve \textbf{agaea} $\emptyset$-begi-d-u-e//
\glb I man he him-hit-\Pst{}-1\Sg{}-\Ind{}//
\glft `I hit the man.' \\{\citep[18f.]{renck1975}}//
\endgl
\xe

\pex\a
\begingl
\gla Ovu-\textbf{da} ma-lo' bei-d-u-e//
\glb Ovi-I this-\Loc{} live-\Pst{}-1.\Sg{}-\Ind{}//
\glft `I, Ovu, am here.'//
\endgl
\a
\begingl
\gla a-\textbf{tata} e-d-a'-e//
\glb woman-they.\Du{}{} come-\Pst{}-3.\Du{}-\Ind{}//
\glft `The two women came.'\\{\citep[19]{renck1975}}//
\endgl
\xe

See \citet[17f., 166, 181]{renck1975}.

\subsubsection{Amele (amel1241), \gls{tng}, Mabuso}

\ex \begingl
\gla {}[{Dana} {ben} {eu} \textbf{age}] ho-ig-a.//
\glb man big that 3\Pl{} come-3\Pl{}-\Todpst{}//
\glft `Those leaders (big men) came.'\\{\citep[after][210, (283)-(284)]{roberts1987amele}}//
\endgl
\xe

See \citet[162, 201, 209f.]{roberts1987amele}.

\subsubsection{Kobon (kobo1249), \gls{tng}, Kalam-Kobon}

\pex
\a
\begingl
\gla Juab Minöp \textbf{kal\textbari{}p} ñ-öb.//
\glb Juab Minöp \Obj.3\Du{}{} give-\Prf.3\Sg{}//
\glft `He gave it to Juap and Minöp.'\\{\citep[after][108, (264)]{davies1989}}//
\endgl
\a
\begingl
\gla \textbf{Yad} Kaunsol nibi b\textbari{} abad aij g\textbari{}-m\textbari{}d-pin.//
\glb 1\Sg{} councillor woman man look.after good do-\Habit-\Pfv.1\Sg{}//
\glft `As councillor I look after the people well.' \\\citep[157, (408ad)]{davies1989}//
\endgl
\xe 

See  \citet[107f., 157]{davies1989} for instances of prenominal pronouns as well as postnominal ``pronominal copies''.

\subsubsection{Usan (usan1239), \gls{tng}, North Adelbert} 

\pex
\a \begingl
\gla eâb igim-ine ne tain \textbf{wo} yâ-nâmb wogub\ldots//
\glb cry.\Ss{} be-1\Sg.\Ds{} and father he me-hit.\Ss{} cease.\Ss{}//
\glft `I was crying and my father he hit me and then\ldots' \\
\citep[after][167, (99)]{reesink1987}//
\endgl
\a 
\begingl
\gla \ldots{} in bo \textbf{an} wau moi qomon gâs ende ig-oun//
\glb {} we again you child unmarried custom like thus be-1\Pl.\Prs{}//
\glft `\ldots we in turn live like the customs of you young men.'\\
\citep[part of][190f., (41)]{reesink1987}//
\endgl
\xe 

See \citet[53f., 167, 190f., 353]{reesink1987}.


\subsubsection{Adang (adan1251), \gls{tng}, Alor-Pantar}

\ex 
\begingl
\gla Sa [Bain \textbf{\textglotstop{}ari}]\unt{NP} b$\epsilon$h.//
\glb 3\Sg.\Sbj{} Bain 3.\Obj{} hit//
\glft `S/he hit Bain.'\\\citep[261, (165)]{robinsonhaan2014}//
\endgl
\xe

\subsubsection{Kaera (kaer1234), \gls{tng}, Alor-Pantar}

\ex 
\begingl
\gla Ilwang \textbf{gang} user\texttildelow{}user bir bleling g-om mi eser-o.//
\glb Ilwang 3\Sg{} \Redup\texttildelow{}quickly run open 3\Sg{}-inside \Loc{} exit-\Fin{}//
\glft `Ilwang quickly ran outside.' (lit. `\ldots ran out to (the) open's inside')\\\citep[129, (98)]{klamer2014}//
\endgl
\xe

\newpage

\subsubsection{Kamang (kama1365), \gls{tng}, Alor-Pantar}

\ex
\begingl 
\gla almakang=ak \textbf{gera}//
\glb people=\Def{} 3.\Contr{}//
\glft `the \{specific group of\} people'\\{\citep[313f., (58a)]{schapper2014}}//
\endgl
\xe

See \citet[313f.]{schapper2014}.

\subsubsection{Sawila (sawi1256), \gls{tng}, Alor-Pantar}

\ex \begingl
\gla {}[aning du \textbf{girra}]\unt{A} [parra]\unt{P} laata//
\glb \Nfin.person \Pl{} 3.\Aarg{} field burn//
\glft `People are burning fields.'\\ \citep[392, (102f)]{kratochvil2014}//
\endgl
\xe

See \citealp[391f.]{kratochvil2014}.
 
\subsubsection{Wersing (wers1238), \gls{tng}, Alor-Pantar}

\ex \begingl
\gla aning \textbf{gnuk} unan le-wena//
\glb person 3.\Du{} louse \Appl-search//
\glft `Those two people searched for lice.'\\\citep[472, (75)]{schapperhendery2014}//
\endgl
\xe

See \citet[472]{schapperhendery2014}.


\subsubsection{Western Pantar (lamm1241), \gls{tng}, Alor-Pantar}

\ex 
\begingl
\gla {}[[Tabang alaku Duinni Maggangkala]\unt{NP} [\textbf{ging}]\unt{NP}]\unt{NP} a-raung yattu ga-ung misingup.//
\glb slave two Duinni Maggangkala 3\Pl.\Act{} INCP-climb tree 3\Sg-head sit//
\glft `The two slaves Duinni Maggangkala climbed up and sat in the tree.'\\\citep[53, (105)]{holton2014}//
\endgl
\xe 

\citet[53, fn. 3]{holton2014} notes that ``Duinni Maggagkala is a single (binomial) name give [sic] to the pair of slaves together.''
Further see \citet[53f.]{holton2014}.


%%%%%%%%%%%%%%%%%%%%%%%%%%%%%%%%%
\subsection{Sino-Tibetan languages}


\subsubsection{East Geshiza (gesh1238), Burmo-Qiangic} 

\ex 
\begingl
\gla rd\textctz{}\ae{} \textbf{lm\ae{}=\textltailn{}\textschwa}=t\super{h}\textschwa~t\super{h}\textschwa{} mp\super{h}ri v-s\super{h}\ae{}=b\textopeno{}, rd\textctz{}\ae{}. b\ae{} \ng\ae{}=\textltailn{}\textschwa{}=t\super{h}\textschwa{} mp\super{h}ri mi-s\super{h}o\ng{}.//
\glb Chinese 3=\Pl.\Abs{}=\Top~\Redup{} snake.\Abs{} \Inv-kill.\Npst.3=\Mod{} Chinese.\Abs{} Tibetan 1=\Pl.\Abs{}=\Top{} snake.\Abs{} \Neg{}-kill.\Npst.1\Pl{}//
\glft `The (Han) Chinese, they kill snakes. [\ldots] We Tibetans do not kill snakes.'\\\citep[480, (7.48)]{honkasalo2019}//
\endgl
\xe

See \citet[388, 400, 438, 480, 507, 646]{honkasalo2019} for further examples.

\subsubsection{Mandarin (mand1415), Chinese}


\ex \begingl
\gla congming de \textbf{ni-men} ziji xiang banfa jiejue ba!//
\glb smart \Lnk{} you-\Pl{} yourselves think ways solve \Exclam{}//
\glft `you smart people think of a way to solve it yourselves.'\\\citep[after][(53a)]{boskovichsieh2013}//
\endgl
\xe 

See \citet[sec. 7.3]{boskovichsieh2013} for further examples.


%%%%%%%%%%%%%%%%%%%%%%%%%%%%%%%%%%
\subsection{Uralic languages}


\subsubsection{Hungarian (hung1274), Ugric}

\ex \begingl
\gla \textbf{Ti} orvos-ok sok-at dolgoz-tok.//
\glb you.\Pl{} doctor-\Pl{} much-\Acc{} work-\Indf.2\Pl//
\glft `You doctors work a lot.' \\\citep[269, (492)]{keneseietal1998}//
\endgl
\xe

See \citet[269]{keneseietal1998} and \citealp[559]{hoehn2015unagr}.

\newpage

\subsubsection{Finnish (finn1318), Finnic}

\ex \begingl
\gla \textbf{Me\textbf{}} naiset menemme nyt saunaan.//
\glb we woman.\Pl{} go.1\Pl{} now sauna.\Ill//
\glft `We women will go to the sauna now.'\\\citep[277, (1335)]{sulkalakarjalainen1992}//
\endgl
\xe

See also \citet[24f.]{hoehn2020ThirdGap}.


%%%%%%%%%%%%%%%%%%%%%%%
\subsection{Creoles}


\subsubsection{Ndyuka (ndyu1242), Creoles}

\ex \begingl
\gla \textbf{u} gaanman fu den liba//
\glb 1/2\Pl{} chief for the.\Pl{} river//
\glft `you chiefs of the rivers'\\\citep[467, (2075)]{huttarhuttar1994}//
\endgl
\xe

See \citet[224, 460, 466f.]{huttarhuttar1994}.

\subsubsection{Nigerian Pidgin (nige1257), Creoles}

\ex \begingl
\gla \textbf{Unà} onyìbo pipul no dè chu kola àt\b{ô}l.//
\glb you.\Pl{} white people \Neg{} \Ipfv{} chew kola \Neg.\Emph{}//
\glft `You white people don't chew kola nut at all.'\\\citep[after][181, (802)]{faraclas1996}//
\endgl
\xe

See \citet[178, 181]{faraclas1996}.

\subsubsection{Kristang (mala1533), Creoles}

\ex
\begingl
\gla  kora jenti muré, tudu \textbf{nus} kristáng bai//
\glb when person die all 1\Pl{} {} go//
\glft `When people die, all we Kristangs go (to the wake).' \\\citep[86, (11)]{baxter1988}//
\endgl
\xe

See \citet[86]{baxter1988}.


%%%%%%%%%%%%%%%%%%%%%%%%%%%%%%%%
\subsection{Various}


\subsubsection{Mi'kmaq (mikm1235), Algonquian}

\ex
\begingl
\gla \textbf{ninen} elnui-yek//
\glb we.\Excl{} people-1\Pl.\Excl//
\glft `we First Nation people'\\
\citep[188]{pacifiqueetal1990}//
\endgl
\xe

Thanks to Watson Williams for help with the glossing.
See \citet[188]{pacifiqueetal1990} for some further (unglossed) examples.

\subsubsection{Katu (nucl1297), Katuic}

\pex
\a \begingl
\gla manuih \textbf{yi}//
\glb people we//
\endgl
\a 
\begingl
\gla \textbf{yi} manuih//
\glb we people//
\glft `we people'//
\endgl
\a
\begingl
\gla \textbf{yi} adi anó \textbf{yi}//
\glb we older.brother younger.brother we//
\glft `we older and younger brothers'\\
\citep[28, (35--37)]{costello1969}//
\endgl
\xe

See \citet[28]{costello1969}.

\subsubsection{Wari' (wari1268), Chapacura-Wanham}

\ex \begingl
\gla \textbf{Wirico} Mon' co pa' na mijac//
\glb \Emph:3\Sg.\M{} \M:name \Infl:\M/\F.\Real.\Pst/\Prs{} kill 3\Sg:\Real.\Pst/\Prs{} pig//
\glft `It was Mon' who killed a pig.'\\\citep[after][303, (570a)]{everettkern1997}//
\endgl
\xe

See \citet[303, 310]{everettkern1997}.

\newpage

\subsubsection{Chitimacha (chit1248), Chitimacha}

\ex \textbf{\textglotstop{}u\v{s}} pan\v{s}' ha hananki' namkinada'\\
`We people who live in this house.'\\
\citep[333]{swadesh1967}\xe

See \citet[333]{swadesh1967}.

\subsubsection{Lezgian (lezg1247), Lezgic}

\ex 
\begingl
\gla \textbf{\v{C}a-z} \v{c}uban-r.i-z, wun har näni-q$^h$ k'wal.i-z q$^h$ifi-zwazj-di q$^h$sam \v{c}i-zwa//
\glb we-\Dat{} shepherd-\Dat{} you:\Abs{} every night-\Postess{} house-\Dat{} return-\Ipfv-\Ptcp-\Nmlz good know-\Ipfv//
\glft `We shepherds know well that you go home every night.' \\\citep[after][259, (682a)]{haspelmath1993}//
\endgl
\xe

See \citet[259]{haspelmath1993}.

\subsubsection{Abkhaz (abkh1244), Northwest Caucasian}

\ex 
\begingl
\gla \textbf{\textcrh{}a(rà)} (\textbf{\v{s}$^o$a(rà)}, \textbf{darà}) a-bà\textcrh\v{c}-aa-ja-y$^o$-c$^o$a//
\glb we you they \Art-garden-\Prev-tend-\Aarg{}-\Pl//
\glft `we (you, they) gardeners' \\\citep[159]{hewitt1989}//
\endgl
\xe

See \citet[157, 159]{hewitt1989}.

\subsubsection{Basque (basq1248), Basque}

\pex
\a
\begingl
\gla Galdu didazue aita-seme-\textbf{ok} afari-ta-ko gogo guzti-a.//
\glb spoil 3\Sg{}.\Abs{}.\Aux{}.1\Sg{}.\Dat{}.\textbf{2\Pl{}.\Erg} father-son-\Proxart.\Pl{}.\Erg{} dinner-\Loc{}-\Lnk{} appetite all-\Det{}.\Abs{}//
\glft `{You, father and son}, have spoiled my whole appetite for dinner.' \\{\citep[502, (90a)]{derijk2008}}//
\endgl 
\a
\begingl
\gla Zor berri-a dugu euskaldun-\textbf{ok} Orixe-rekin.//
\glb debt new-\Det{}.\Abs{}{} 3\Sg{}.\Abs{}.\Aux{}.\textbf{1\Pl{}.\Erg} Basque-\Proxart.\Pl{}{} Orixe-\Com{}//
\glft `{We Basques} have a new debt to Orixe.' \\{\citep[502, (91a)]{derijk2008}}//
\endgl
\xe

See \citet[210]{saltarelli1988}, \citet[122]{trask2003}, \citet[482, 501f.]{derijk2008}, \citet[67]{areta2009} and especially \citet{artiagoitia2012DP}.

\subsubsection{Classical Nahuatl (clas1250), Aztecan}

\pex
\a
\begingl
\gla \textbf{Ni}cu\={\i}ca \textbf{ni}Petoloh.//
\glb I-sing I-am-Peter//
\glft `I, Peter, sing.'//
\endgl
\a
\begingl
\gla \textbf{N\={e}ch}itta \textbf{ni}Petoloh.//
\glb he-sees-me I-am-Peter//
\glft `He sees me, Peter.'\\{\citep[193]{andrews1975}}//
\endgl
\a
\begingl
\gla \textbf{No}cal \textbf{ni}Petoloh.//
\glb It-is-my-house/they-are-my-houses I-am-Peter//
\glft `It is my house (and I am Peter)./They are my houses (and I am Peter).'\\{\citep[194]{andrews1975}}//
\endgl
\xe

See \citet[192--194]{andrews1975} and \citealp[ch. 17.3]{andrews2003}.

\subsubsection{Evenki (even1259), Tungusic}

\ex \begingl
\gla  \textbf{Bu} bejumimni-l eme-re-$\emptyset$.//
\glb we hunter-\Pl{} come-\Nfut-3\Pl//
\glft `We, hunters, came.' \\\citep[199, (794)]{nedjalkov1997}//
\endgl
\xe

See \citet[197, 199]{nedjalkov1997}.



\subsubsection{Hixkaryana (hixk1239), Cariban}

\citet[131]{derbyshire1979} suggests that Hixkaryana does not have integrated adnominal person marking (and no adnominal demonstratives), but instead paratactic constructions like (\nextx).

\pex
\a
\begingl
\gla m\textbari{}nayar\textbari{} hor\textbari{} \textbf{amna} ntono. n\textbari{}mno + hokono rma \textbf{amna}//
\glb species.of.leaf seeking we.\Excl{} went house {one.occupied.with} same-ref we.\Excl{}//
\glft `We housebuilders went looking for leaves'\\{\citep[131, (290)]{derbyshire1979}}//
\endgl
\a
\begingl
\gla nux mokro raheno//
\glb my.younger.brother that.one he.seduced.me//
\glft `That younger brother of mine seduced me'\\{\citep[132, (293a)]{derbyshire1979}}//
\endgl
\xe



\subsubsection{Japanese (nucl1643), Japanese} 

\ex \begingl
\gla \textbf{wareware} nihonjin//
\glb we Japanese//
\glft `we Japanese'\\\citep[254]{hinds1988}//
\endgl
\xe

See \citet[254, 261]{hinds1988} and for more detailed discussion with different theoretical proposals: \citet[780]{noguchi1997}, \citet[sec. 3.2]{furuya2008} and \citet{inokuma2009}.

\subsubsection{Korean (kore1280), Korean}


\ex \textbf{wuli-(tul)} hankwuk salam\\
 `we Koreans'\\
 \citep[292]{sohn1994}
 \xe
 
\pex
\a \begingl
\gla \textbf{wuli} ttokttokhan enehakcatul//
\glb we smart linguists//
\endgl
\a \begingl
\gla ttokttokhan \textbf{wuli} enehakcatul//
\glb smart we linguists//
\glft `we smart linguists'\\\citep[151, (15)]{choi2014phd}//
\endgl
\xe

See \citet[284; 292]{sohn1994} and \citet[151--154]{choi2014phd}.


\subsubsection{Kalaallisut/West Greenlandic (kala1399), Eskimo}


\pex
\a
\begingl
\gla kalaalli-t \textbf{uagut}//
\glb Greenlander-\Abs{}.\Pl{} we//
\glft `we Greenlanders'\\{\citep[after][110; gloss extrapolated]{fortescue1984}}//
\endgl
\a
\begingl
\gla \textbf{uagut} kalaali-u-sugut//
\glb we Greenlander-be-1\Pl{}.\Ptcp{}//
\glft `we Greenlanders' \\{\citep[after][257]{fortescue1984}}//
\endgl
\xe

See \citet[110, 253, 256f.]{fortescue1984}.

\subsubsection{Lakkia (lakk1238), Kadai}

\ex 
\begingl
\gla \textbf{tau$^{51}$} hou$^{24}$ {\textglotstop{}at$^{55}$ jen$^{11}$ kjã:u$^{24}$}//
\glb 1\Pl{} two sister//
\glft `we, the two sisters'\\\citep[137, (40)]{fan2019}//
\endgl
\xe

\subsubsection{Turkish (nucl1301), Turkic}

\ex 
\begingl
\gla \textbf{biz} Türk-ler vatan-{\i}m{\i}z-{\i} sev-er-iz//
\glb we Turk-\Pl{} mother.country-1\Pl-\Acc{} love-\Aor-1\Pl{}//
\glft `We Turks love our country'\\\citep[297, (1075)]{kornfilt1997}//
\endgl
\xe

See \citet[288, 297f.]{kornfilt1997}.


\subsection*{Abbreviations}

\printglosses 


\bibliographystyle{unifiedNOURLNODOI-noampersand}
\bibliography{S3-literature}

\end{document}


